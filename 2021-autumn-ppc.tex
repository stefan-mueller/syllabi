\documentclass[abstract=on,parskip=full,headings=standardclasses,fontsize=11pt,paper=a4]{scrartcl}
%\usepackage[paper=a4paper,left=21mm,right=21mm,top=25mm,bottom=25mm]{geometry}
\usepackage[paper=a4paper,left=20mm,right=15mm,top=25mm,bottom=25mm]{geometry}

\usepackage[utf8]{inputenc}
\usepackage[T1]{fontenc}
\usepackage[english]{babel}
\usepackage{soul}

\usepackage{adjustbox}
%\usepackage{amsmath}
\usepackage{graphicx}
%\usepackage{fullpage}
\usepackage{authblk}
\usepackage{setspace}
\usepackage{caption}
\usepackage{booktabs}
\usepackage{url}
\usepackage{comment}
\urlstyle{sf}
\usepackage{lmodern}
\usepackage[parfill]{parskip}
%\usepackage{url}
%\urlstyle{same}
\usepackage[small]{titlesec}
\usepackage{marvosym}

\setcounter{secnumdepth}{0}

\addto\captionsenglish{% Replace "english" with the language you use
  \renewcommand{\contentsname}%
    {Course Structure}%
}



%\deffootnote[10pt]{10pt}{10pt}{\makebox[15pt][l]{\thefootnotemark\hspace{10pt}}}

% Use authoryear-comp to create: (Müller 2015, 2016) instead of (Müller 2015; Müller 2016)

% posscite function

\usepackage[style=authoryear-comp,
    maxcitenames=99,
    maxbibnames=99,
    doi=false,
    %sorting=ynt,
    firstinits=false,
    isbn=false,
    date=short,
    dashed=false,
    url=false,
    sortcites=false,
    backend=bibtex]{biblatex}

\makeatletter
\def\blx@maxline{77}
\makeatother


\DeclareNameFormat{labelname:poss}{% Based on labelname from biblatex.def
  \nameparts{#1}% Not needed if using Biblatex 3.4
  \ifcase\value{uniquename}%
    \usebibmacro{name:family}{\namepartfamily}{\namepartgiven}{\namepartprefix}{\namepartsuffix}%
  \or
    \ifuseprefix
      {\usebibmacro{name:first-last}{\namepartfamily}{\namepartgiveni}{\namepartprefix}{\namepartsuffixi}}
      {\usebibmacro{name:first-last}{\namepartfamily}{\namepartgiveni}{\namepartprefixi}{\namepartsuffixi}}%
  \or
    \usebibmacro{name:first-last}{\namepartfamily}{\namepartgiven}{\namepartprefix}{\namepartsuffix}%
  \fi
  \usebibmacro{name:andothers}%
  \ifnumequal{\value{listcount}}{\value{liststop}}{'s}{}}
\DeclareFieldFormat{shorthand:poss}{%
  \ifnameundef{labelname}{#1's}{#1}}
\DeclareFieldFormat{citetitle:poss}{\mkbibemph{#1}'s}
\DeclareFieldFormat{label:poss}{#1's}
\newrobustcmd*{\posscitealias}{%
  \AtNextCite{%
    \DeclareNameAlias{labelname}{labelname:poss}%
    \DeclareFieldAlias{shorthand}{shorthand:poss}%
    \DeclareFieldAlias{citetitle}{citetitle:poss}%
    \DeclareFieldAlias{label}{label:poss}}}
\newrobustcmd*{\posscite}{%
  \posscitealias%
  \textcite}
\newrobustcmd*{\Posscite}{\bibsentence\posscite}
\newrobustcmd*{\posscites}{%
  \posscitealias%
  \textcites}

\renewbibmacro{in:}{} % no "in" before article

\renewcommand*{\bibpagespunct}{\addcomma\space} % ":" instead of pp
\DeclareFieldFormat{pages}{#1}

% Colon after title
\renewcommand{\subtitlepunct}{\addcolon\addspace }

% Colon instead of pp in references
\renewcommand*{\bibpagespunct}{\addcolon\space}
\DeclareFieldFormat{pages}{#1}

% Colon after name in text
\renewcommand*{\postnotedelim}{\addcolon\space}
\DeclareFieldFormat{postnote}{#1}
\DeclareFieldFormat{multipostnote}{#1}

% Remove brackets around year in bibliography
\usepackage{xpatch,filecontents}

\xpatchbibmacro{date+extrayear}{%
  \printtext[parens]%
}{%
  \setunit*{\addperiod\space}%
  \printtext%
}{}{}

% Supresses URL accessed day
\AtEveryBibitem{%
  \ifentrytype{electronic}
    {}
    {\clearfield{urlyear}\clearfield{urlmonth}\clearfield{urlday}}}
%\DefineBibliographyStrings{english}{%
%urlseen = {Accessed},}

\renewbibmacro*{volume+number+eid}{% number of journal in brackets
 \printfield{volume}%
  %\setunit*{\adddot}% DELETED
  \setunit*{\addnbthinspace}% NEW (optional); there's also \addnbthinspace
  \printfield{number}%
  \setunit{\addcomma\space}%
  \printfield{eid}}
\DeclareFieldFormat[article]{number}{\mkbibparens{#1}}



% Change edition field

\DeclareFieldFormat{edition}%
                   {\ifinteger{#1}%
                    {\mkbibordedition{#1}\addthinspace{}edition}%
                    {#1\isdot}}

% New command to show doi, or url or isbn or issn field
% http://tex.stackexchange.com/questions/48400/biblatex-make-title-hyperlink-to-dois-url-or-isbn
\newbibmacro{string+doiurlisbn}[1]{%
  \iffieldundef{doi}{%
    \iffieldundef{url}{%
      \iffieldundef{isbn}{%
        \iffieldundef{issn}{%
          #1%
        }{%
          \href{http://books.google.com/books?vid=ISSN\thefield{issn}}{#1}%
        }%
      }{%
        \href{http://books.google.com/books?vid=ISBN\thefield{isbn}}{#1}%
      }%
    }{%
      \href{\thefield{url}}{#1}%
    }%
  }{%
    \href{http://dx.doi.org/\thefield{doi}}{#1}%
  }%
}

% Necessary to remove dot after question mark in title
%\newcommand{\killpunct}[1]{}    

% Make full stop after title and before quotation marks in title field
\DeclareFieldFormat{title}{\usebibmacro{string+doiurlisbn}{\mkbibemph{#1}}}
\DeclareFieldFormat[article,incollection,unpublished,phdthesis]{title}%
    {\usebibmacro{string+doiurlisbn}{\mkbibquote{#1}}}
   % {\usebibmacro{string+doiurlisbn}{\mkbibquote{#1.\isdot}}}

\renewcommand*{\newunitpunct}{.\space}


\bibliography{/Users/smueller/Documents/GitHub/literature/muellerlibrary.bib}
%\bibliography{/Users/stefan/GitHub/literature/muellerlibrary.bib}


\usepackage{xcolor}
%\definecolor{JournalBlue}{RGB}{0, 12, 146}

\definecolor{JournalBlue}{RGB}{25, 63, 144}

%https://en.wikibooks.org/wiki/LaTeX/Colors
\usepackage[colorlinks=true, linkcolor=JournalBlue, filecolor=black, urlcolor=JournalBlue, pdfborder={0 0 0},citecolor=JournalBlue]{hyperref}%RoyalBlue
%\usepackage[colorlinks]{hyperref}

\clubpenalty = 10000 
\widowpenalty = 10000 
\displaywidowpenalty = 10000

\setlength\parindent{0pt}


\usepackage{titlesec}
\titleformat{\section}
   {\normalfont\Large\bfseries}{\thesection}{1em}{}

   

\begin{document}
	

\singlespacing

\noindent
\adjustbox{valign=t}{\begin{minipage}{0.1\textwidth}% adapt widths of minipages to your needs
\includegraphics[width=\linewidth]{pictures/ucd_logo}
\end{minipage}}%
\hfill%
\adjustbox{valign=t}{\begin{minipage}{0.9\textwidth}\raggedleft
{%\footnotesize
\textbf{Stefan Müller, PhD} \\
Assistant Professor and Ad Astra Fellow \\
School of Politics and International Relations\\
%Connected\_Politics Lab \\
University College Dublin \\
Belfield, Dublin 4, Ireland \\
%\Telefon\ + 353\,89\,975\,25\,79 \\
\Letter\ \href{mailto:stefan.mueller@ucd.ie}{\textsf{stefan.mueller@ucd.ie}}\\
\ComputerMouse\ \url{https://muellerstefan.net} \\
}
\end{minipage}}

\singlespacing
\vspace{1cm}

\begin{center}
{\large Level 3 Module; 
Autumn Trimester 2021} \\ 
\bigskip

{\Large \textbf{Parties and Party Competition} (\href{https://hub.ucd.ie/usis/!W_HU_MENU.P_PUBLISH?p_tag=MODULE&MODULE=POL30720}{POL30720})} 
\bigskip


{\large  \textcolor{black}{Draft (Version: \today)}}\\
\bigskip

Latest version at: \url{https://muellerstefan.net/teaching/2021-autumn-ppc.pdf}
\end{center}

\vspace{1.5cm}



\hrule
\medskip
% first column
\begin{minipage}[t]{0.5\textwidth}
Time: Tuesday 11:00--11:45 \& Thursday 11:00--11:45
%Time: \\
%Locations:  
\begin{itemize} \renewcommand\labelitemi{--} \small 
\item Tue: \href{https://goo.gl/maps/vCQP95QtKrTWGDMx7}{E2.16-SCE (O'Brien Science Centre East)}
\item Thu: \href{https://goo.gl/maps/gKF6596m1vvxFeeHA}{B003-CSI (Computer Science Centre)} 
\end{itemize}
ECTS: 10 \\
Format: Lecture; in-class discussions
\end{minipage}
%second column
\begin{minipage}[t]{0.49\textwidth}
\begin{flushright}
Module coordinator: Stefan Müller, PhD \\
 \href{mailto:stefan.mueller@ucd.ie}{\textsf{stefan.mueller@ucd.ie}} \\
 \url{https://muellerstefan.net} \\
Office:  Newman Building, G312 \\
Office hours: Tue 13:00--14:00
\end{flushright}
\end{minipage}

\medskip
%\vspace{2.5mm}
\hrule 

\section*{Module Description}

How can we identify differences between party systems, determine party positions, and measure public opinion? Do parties keep their promises, or are politicians ``pledge breakers''? Are promises in certain policy areas more likely to be fulfilled? In what policy areas do parties differ in terms of their positions and issue emphasis? And do parties respond to changes in public opinion?

In this module, we first discuss the main functions of political parties, outline features of representative democracies, and identify ways of measuring public opinion. Next, we assess whether parties keep their promises, whether the ``mandate model of democracy'' is a desirable and realistic mode of political representation, and how we could improve existing studies on election pledge fulfilment. Afterwards, we investigate parties' willingness and capacity to respond to changes in public opinion. Fourth, we discuss different approaches to measuring party positions, political ideology, and the salience of policy areas. Based on these methodological approaches, we identify the circumstances under which parties change their positions and issue emphasis. Finally, we briefly discuss alternative types of political participation that go beyond representative government and electoral democracy.




\section*{Learning Outcomes}

\begin{enumerate}
\item Extensive knowledge of central theories of representation,  the mandate model of democracy, and party competition
\item Detailed insights into past and current approaches to study questions about pledge fulfilment, party positions, responsiveness, and issue ownership 
\item Critical reading and discussing  complex academic literature and diverse  quantitative and qualitative methodological approaches
\end{enumerate}

\section*{Indicative Module Content}

The following topics will be covered in this course: parties and party systems; the ``mandate model of democracy''; measuring and aggregating public opinion; economic voting; the cost of governing; responsiveness; party competition; party positions, salience, and issue ownership; campaign pledges; parties' online communication and campaigning

\section*{Approaches to Teaching and Learning}

\begin{itemize}
\item Active and task-based learning
\item Group work and discussions
\item In-class debates
\item Problem-based learning
\end{itemize}




%\section*{Module Requirements}


\subsection*{Overview of Assessment}

\begin{itemize}
\item 1,000 word response paper: 20\% 
\item Descriptive data analysis: assess and interpret data on party competition in Europe (1,000 words): 20\%
\item 2,500--3,000 word essay from a choice set of questions: 60\%
\end{itemize}


\subsection*{Expectations and Guidelines}
\begin{itemize}


\item You are expected to read the papers or chapters assigned under \textbf{mandatory readings}. These readings serve as the basis for in-class discussions.% about the advantages, disadvantages, and applicability of the various approaches to social science questions. 

\item You submit one \textbf{response paper} (1,000 words) throughout the course, which counts towards 20\% of the grade. By Week 3, everyone will have been assigned a week where they will prepare a response paper. Response papers must be submitted via \href{https://brightspace.ucd.ie/}{Brightspace} no later than \textbf{Tuesday, 9am of the respective week}, meaning that the assignment has to be submitted \textit{before} the texts are discussed in class. You are required to choose \textit{one} of the \textit{required or suggested} readings for that week (readings marked with a star may not be used).  Response papers must contain the following two aspects:
\begin{enumerate}
\item Identify either a limitation of the paper (e.g., how a variable is measured, or an unreasonable/unnecessary assumption) or a possible extension. Either way, you should have only one argument in these papers. 
\item Suggest a possible solution to that limitation or describe how you would carry out the extension. Note that what you propose should be feasible (ideally, you should be able to conduct the suggested research proposal). If, for example, you find the author's data weak, then you should identify better data, or at least propose a plausible way of collecting these data. If you think the method is wrong, explain why and suggest a better one. If the conclusions do not follow from the premises, discuss what conclusions are actually supported. A specific course of action should be outlined. I am not interested in a summary of the selected paper. The idea is to develop innovative ideas for future research projects. 
\end{enumerate}

\item You also submit a \textbf{descriptive data analysis}. Based on a new interactive collection of materials and data about party competition in Europe, you will explore an empirical question by using existing tools to interpret quantitative data. You do not require prior knowledge of coding or statistical programming languages. The descriptive data analysis counts toward 20\% of the final grade. Questions will be allocated in Week 3 of the course and more details on the expectations will be provided in class and on \href{https://brightspace.ucd.ie}{Brightspace}. The descriptive data analysis should not exceed 1,000 words.

\item Finally, you submit an \textbf{essay}  which counts towards 60\% of the final grade. The essay must not exceed 2,500--3,000 words  (including bibliography, captions, and footnotes) and  will tackle one of the `discussion questions' listed below. For this essay, you are required to (i) draw on academic literature (articles and/or books) and (ii) properly cite the academic literature you use to prepare your essay, focusing on \href{https://ooir.org/journals.php?category=polisci}{peer-reviewed journals from political science}. You should read beyond the reading list for this essay and attach an \textit{alphabetised}  bibliography to your essay. The essay must be submitted via \href{https://brightspace.ucd.ie}{Brightspace} as a \texttt{PDF} document before \textbf{17 December 2021 (8:00pm CET)}. More information on the essay will be provided in the seminar. For information on academic writing, I recommend the following sources:

\begin{itemize}
\item \fullcite{dunleavy14}.
\item \fullcite{heard16}.
\end{itemize}

For the essay, I recommend to pay special attention to the following aspects:

\begin{itemize}
\item \textit{Focus on argumentation, demonstrate critical thinking}: Your essay will be judged primarily on your ability to make nuanced arguments and to demonstrate your understanding of the nuances of the arguments presented by the authors discussed in the course and readings that go beyond the syllabus. While you are expected to engage with the material in the course during your essay, a good essay will do so in a creative way where your own voice comes through clearly. This can be done by critically commenting on the arguments of others; creatively combing arguments from others to make a case; and/or presenting your own original arguments in attempting to improve upon shortcomings in the literature that you have identified.
\item \textit{Read deeply, read widely}: Reading deeply is the most important thing for developing your essay. \textbf{But you should also read widely, consulting sources both within and beyond the syllabus.} It is possible to write a great paper by focusing on just a small number of sources. But this is rare enough. As a rule of thumb, well-researched papers usually average between one and two distinct references per double-spaced page. For a 2,500--3,000-word essay, this will amount to approximately 10--15 distinct references to texts that you have read and analysed closely. 
\item \textit{Presentation}: Be attentive to the presentation of your essay, including consistent referencing-style (with page numbers provided), a bibliography, and  a consistent layout. Learning how to deliver well-presented and polished-looking work is part of your undergraduate training and a highly transferable skill. Take it seriously. Poor presentation will result in lost marks. If you require information on proper citation style, please refer to the guidelines of the American Political Science Association: 
\begin{itemize}
\item \fullcite{apsa18}.
\end{itemize}
\end{itemize}
\end{itemize}


\subsection*{Essay Questions}

Please choose \textbf{one} of the three questions below and make sure to follow the essay guidelines described above. 

\begin{enumerate}

%\item dealignment question page 76 Is dealignment a healthy development for democracy, or is it a trend we should be concerned about? DESCIRBE DAELIGNEMNT, PROVEIDE EXMPLES

\item Are governments in which parties keep higher percentages of their previous campaign pledges more democratic than governments in which parties keep fewer pledges? Discuss the reasons for positive \textit{and} negative answers to this question.

\item Political parties have transformed  over time and new types of party organizations have emerged \autocite{katz95,katz09}. By referring to the academic literature and by providing examples answer the following questions: Which democratic functions of political parties have declined? And which democratic functions have been better fulfilled by modern political parties.
%\item \posscite[5]{katz95} cartel party thesis argues that ``parties [have] become agents of the state and employ the resources of the state (the party state) to ensure their own collective survival.'' Yet, \textcite[507]{koole96} states that ``the reality of Western party systems does not show an effective cartel of parties.'' First, outline the reasons for these different conclusions. Second, assess whether the developments described in \textcite{katz95} threaten representative democracy.



%NEW QUESTION ON GOVERNMENT VS VOTING

%item  %Parties sometimes Why do parties enter a government as a smaller coalition partner? From a normative and strategic perspective, should parties remain in opposition rather than functioning as a junior coalition partner? Your answer should include references to the literatures on coalition formation, voting behaviour, and the cost of governing.


%NEW QUESITON ON RESPONSIVENESS

\item Does retrospective performance voting provide a useful mechanism of holding political parties accountable, or are the fears about `blind retrospection' justified?

%Positions of political parties and voters are often measured on a left-right dimension. Is the left-dimension sufficient to accurately capture party positions and party competition? 

%The concept of responsiveness implies that governments adjust their policies based on changes in citizens' preferences. Summarise different ways of measuring policy responsivness. 

% First, summarise and critically evaluate potential ways of measuring policy responsiveness. Second, explain whether and to what degree parties react to changes in public opinion. Third, discuss whether parties should indeed respond to public opinion or instead stick to their principles.

\end{enumerate}


Plagiarism is an issue we take very serious here in UCD. Please familiarize yourself with the definition of plagiarism on UCD's website and make sure not to engage in it.



\section*{Late Submission Policy}

All written work must be submitted on or before the due dates. Students will lose one point of a grade for work up to 5 working days late ($B-$ becomes $C+$). Students will lose two grade points for work between 5 and 10 working days late ($B-$ becomes $C$). When an extension of more than two weeks is necessary, the student will need to apply for extenuating circumstances application via the SPIRe Programme Office.

\begin{comment}
\section*{Feedback Strategy}

\begin{itemize}
\item Feedback individually to students, post-assessment
%\item Feedback individually to students, on an activity or draft prior to summative assessment
%\item Group/class feedback, post-assessment
\end{itemize}
\end{comment}


\begin{table}[h] \centering \onehalfspacing
\caption*{Student effort hours}
\begin{tabular}{ l r} 
\toprule
Student effort type &  Hours \\
\midrule
Seminars & 22 \\
Autonomous Student Learning  & 196 \\
\textbf{Total} & \textbf{220} \\
\bottomrule
\end{tabular}
\end{table}



\section*{Introductory Readings}

The seminar does not build on a single text book, but relies  on peer-reviewed papers and book chapters. For  a general overview of the course content, I recommend the following books. Note that I do \textit{not} expect you to buy any of these books since the required and optional readings  for this module will be almost exclusively freely available online through your UCD Library account.

\begin{itemize}
\item \fullcite{devries21}.
\item \fullcite{costello21}.
\item \fullcite{powell00}.
%\item \fullcite{dalton11}.
%\item \fullcite{sartori05}.
%\item \fullcite{gallagher11}.
%\item \fullcite{naurin19}.
%\item \fullcite{mair13}.
\end{itemize}




\section*{Technical Background and Prerequisites}

The course requires knowledge of general approaches and theories of political science. The following books provide very good introductions to research design and applied quantitative methods.

\subsection*{Research Design and Quantitative Methods}
\begin{itemize}
%\item \fullcite{king94}.
%\item \fullcite{gerring01}.
\item \fullcite{kellstedt18}.
\item \fullcite{llaudet22}. %(Note: link to book provided on Brightspace)
%\item \fullcite{wickham17}.
\end{itemize}


\section*{Syllabus Modification Rights}

I reserve the right to reasonably alter the elements of the syllabus at any time by adjusting the reading list to keep pace with the course schedule. Moreover, I may change the content of specific sessions depending on the participants' prior knowledge and research interests.


\section*{Additional Covid-19 Guidelines}

Covid-19 continues to pose a threat to our well-being and health. We all need to follow UCD's guidelines, which involves wearing masks in the lecture rooms. I will also wear  a mask at all times. If you come to my office hours in person, please make sure to wear a mask. If you are unwilling or unable to wear a mask, we can meet virtually. If you are not feeling well, stay home! I try to make all relevant materials available to everyone: I live-will record all lectures, provide the slides, and upload all readings. Protecting everyone's health is most important. Should you be sick or need a longer period of absence, please get in touch  and I happily work with you to ensure your success in this module. We are in this together -- let's try our very best in the months to come and support each other.


%\newpage

\tableofcontents

\section{Week 1: Conceptualising Representative Democracy (14 Sept \& 16 Sept 2021)}

\begin{itemize}
\renewcommand\labelitemi{--}
\item Expectations
\item Discussion of syllabus
\item What are the main differences between the majoritarian and proportional visions of democracy?
\end{itemize}


%\hl{formerly slides from week 3}


\subsubsection*{Mandatory Readings}
\begin{itemize}
\item \fullcite[ch. 1--2]{powell00}.

\end{itemize}


\subsubsection*{Optional}
\begin{itemize}
\item \fullcite{katz17}.
\item \fullcite{bulsara09}.
\end{itemize}

%\hl{adjust this and use different lesson from books??!!}


\section{Week 2: Parties and Party Systems (21 Sept \& 23 Sept 2021)}


%\hl{NOTE: De Vries et al, chapter 8.10 (pp 151)}



\begin{itemize}
\renewcommand\labelitemi{--}
\item  What are political parties?
\item How can we classify different types of democracies?
\end{itemize}

\subsubsection*{Mandatory Readings}
\begin{itemize}
%\item \fullcite{boix07}.
\item \fullcite{katz95}
\item \fullcite{mair08}.
%\item \fullcite[Kapitel 1]{dalton11}.
\end{itemize}




\subsubsection*{Optional}
\begin{itemize}
\item \fullcite{katz09}.
\item \fullcite{koelln15}.
\end{itemize}


\section{Week 3:  Party Competition  (28 Sept \& 30 Sept 2021)}

\begin{itemize}
\renewcommand\labelitemi{--}
\item What goals do parties and politicians pursue?
\item What does Lijphart mean by the Westminster Model of Democracy and the Consensus Model of Democracy?
%\item How do parties compete with each other, and how can we measure party competition?
\end{itemize}

\subsubsection*{Mandatory Readings}
\begin{itemize}
\item \fullcite{strom90}.
\item \fullcite[ch. 1--3]{lijphart12}.
%\item \fullcite{adams09b}. % BJPS
\end{itemize}


\subsubsection*{Optional}
\begin{itemize}
\item \fullcite{somertopcu15}.
\item \fullcite{stokes63}.
\item \fullcite{tavits07}.
\item \fullcite{boehmelt16}.
\item \fullcite{muellerregan}.
%\item \fullcite{wagner14}.
%\item \fullcite{spoon15}.
\end{itemize}





%\begin{comment}
\section{Week 4: Governments and Coalitions (5 Oct \& 7 Oct 2021)}


%\hl{RESTRUCTURE AND ADD MORE ON COALITIONS}

%\hl{Note: chapter 10 de vries et al}


\begin{itemize}
\renewcommand\labelitemi{--}
\item How do we distinguish types of government coalitions?
\item Which government types are most frequent across Europe?  What could explain the variation over time and across countries?
\item Can voters accurately predict the government formed after an election?
%\item  What do we mean by political cleavages?
%\item How have political cleavages shaped party competition? 
%\item How do political cleavages shape party competition and policy outcomes?
\end{itemize}


\subsubsection*{Mandatory Readings}

%\hl{potentially revise readings}


\begin{itemize}
%\item \fullcite[ch. 1--3]{lijphart12}.
\item \fullcite[ch. 10]{devries21}.
\item \fullcite{martin01}.
%\item \fullcite{dalton96}.
\end{itemize}



\subsubsection*{Optional}
\begin{itemize}
\item \fullcite{bowler22}.
\item \fullcite{golder06}.
%\item \fullcite{amorimneto97}.
\item \fullcite{powell06}.
%\item \fullcite{golder14}.
\item \fullcite{carey95}.
%\item \fullcite{iversen06}.
\end{itemize}



%Making Votes Count: Strategic Coordination in the World's Electoral Systems, Cambridge, Chapters 1, 2, 3
%Iversen, T. and Soskice, D. (2006). Electoral Institutions and the Politics of Coalitions: Why Some Democracies Redistribute More than Others. American Political Science Review, 100(2):165–181

%\end{comment}



\section{Week 5: Parties in Government and Challenger Parties  (12 Oct \& 14 Oct 2021)}

\begin{itemize}
\renewcommand\labelitemi{--}
\item How do incumbent parties react to challenger parties or new parties?
\item Why do parties join a coalition? 
\item What happens when populist parties enter parliament?
\end{itemize}

\subsubsection*{Mandatory Readings}
\begin{itemize}
\item \fullcite{kluever19}.
\item \fullcite{abouchadi20}.
\end{itemize}


\subsubsection*{Optional}
\begin{itemize}
\item \fullcite{fortunato19}.
\item \fullcite{warwick06}.
%\item \fullcite{ecker17}.
%\item \fullcite{martin01}.
\item \fullcite{bischof20}.
\end{itemize}


%Does winning pay? Electoral success and government formation in 15 West European countries



\section{Week 6:  The Mandate Model of Democracy (19 Oct \& 21 Oct 2021)}


\begin{itemize}
\renewcommand\labelitemi{--}
\item What is the `democratic mandate'? 
\item How can we measure campaign promises/pledges?
\item Do parties fulfil their promises?
\end{itemize}

\subsubsection*{Mandatory Readings}
\begin{itemize}
\item \fullcite{mansbridge03}.
\item \fullcite{thomson17}.
\item \fullcite{mueller20}.
\end{itemize}

\subsubsection*{Optional}
\begin{itemize}
\item \fullcite{manin99}.* (not suitable for response paper)
\item \fullcite{thomson16b}.
\item \fullcite{matthiess20}.
\item \fullcite{thomson18}.
\end{itemize}

\section{Week 7:  READING WEEK}



\section{Week 8:  Measuring Party Positions and Issue Salience (2 Nov \& 4 Nov 2021)}


\begin{itemize}
\renewcommand\labelitemi{--}
\item What are differences between positions, salience, and issue ownership?
\item How can we measure latent policy positions? 
\item What are methodological difficulties when measuring party positions?
\end{itemize}

\subsubsection*{Mandatory Readings}
\begin{itemize}
\item \fullcite{laver14}.
\item \fullcite{budge15}.
\end{itemize}


\subsubsection*{Optional}
\begin{itemize}
\item \fullcite{mikhaylov12}.
\item \fullcite{proksch10}.
\item \fullcite{benoit20}.*  (not suitable for response paper)
\end{itemize}


%\section{Saturday, 9 March--Sunday, 22 March: Reading Weeks}


%\begin{comment}
\section{Week 9: Responsiveness (9 Nov \& 11 Nov 2021)}

\begin{itemize}
\renewcommand\labelitemi{--}
\item What is democratic responsiveness?
\item Do parties and parties and politicians react to public opinion?  And from a normative perspective, should political actors change their positions and policies depending on citizens' preferences?
\end{itemize}

\subsubsection*{Mandatory Readings}
\begin{itemize}
\item \fullcite{soroka18}.
\item \fullcite{kluever16}.
\end{itemize}

\subsubsection*{Optional}
\begin{itemize}
\item \fullcite{wlezien95}.
\item \fullcite{powell04b}.* (not suitable for response paper)
%\item \fullcite{eulau77}.
\item \fullcite{page83}.
\item \fullcite{stimson95}.
%\item \fullcite{soroka10}.
\end{itemize}
%\end{comment}

\begin{comment}
\section{Week 9: Economic Voting and the Cost of Governing (9 Nov \& 11 Nov 2021)}


\begin{itemize}
\renewcommand\labelitemi{--}
\item What is democratic accountability?
\item Why do government parties regularly lose public support?
\end{itemize}

\subsubsection*{Mandatory Readings}

\begin{itemize}
\item \fullcite{anderson00}.
\item \fullcite{muellerlouwerse}.
\end{itemize}

\subsubsection*{Optional}
\begin{itemize}
\item \fullcite{marsh10}.
\item \fullcite{wlezien17b}.
\item \fullcite{herzog15}.
\end{itemize}
\end{comment}
 
\section{Week 10: The (Ir)Rational Voter? (16 Nov \& 18 Nov 2021)}


\begin{itemize}
\renewcommand\labelitemi{--}
\item Are voters rational decision-makers, as assumed in many theories of representation? 
\item Under what circumstances do voters behave (ir)rationally? What are consequences of irrational voting behaviour on political processes and decisions?
\end{itemize}


\subsubsection*{Mandatory Readings}


 \begin{itemize}
\item \fullcite[ch. 5]{achen16}.
\item \fullcite{healy10}.
\item \fullcite{muellerkneafsey}.
\end{itemize}

\subsubsection*{Optional}
\begin{itemize}
\item \fullcite{sances17}.
%\item \fullcite{reif80}.
\item \fullcite{fowler18}.
\item \fullcite{holman22}.
%\item \fullcite{healy13}.
\end{itemize}



\section{Week 11: Participation Beyond Political Parties (23 Nov \& 25 Nov 2021)}

\begin{itemize}
\renewcommand\labelitemi{--}
\item What are problems associated with representative politics?
\item What other forms of participation exist? What are their strengths and weaknesses?
\item How can these alternative forms of participation supplement representative democracy?
\end{itemize}

\subsubsection*{Mandatory Readings}
\begin{itemize}
\item \fullcite{dryzek19}.* (not suitable for response paper)
\item \fullcite{farrell14}.* (not suitable for response paper)
\item \fullcite{parthasarathy19}.
\end{itemize}


\subsubsection*{Optional}
\begin{itemize}
\item \fullcite[ch. 1--3]{altman11}.* (not suitable for response paper)
\item \fullcite{bowler07}.
\item \fullcite{mair13}.* (not suitable for response paper)
\item \fullcite{colombo18}.
\item \fullcite{hug09}.* (not suitable for response paper)
\end{itemize}



\section{Week 12: Political Parties, the Media, and Digital Democracy (30 Nov \& 2 Dec 2021)}


\begin{itemize}
\renewcommand\labelitemi{--}
\item How does the internet change democratic decision making and representation?
\item Do politicians and parties react to online discussions?
\end{itemize}

\subsubsection*{Mandatory Readings}
\begin{itemize}
\item \fullcite{king17b}.
\item \fullcite{guess19}.
\end{itemize}


\subsubsection*{Optional}
\begin{itemize}
%\item \fullcite{farrell12}.
\item \fullcite{foos22}.
\item \fullcite{barbera19}.
\item \fullcite{gilardi21a}.
%\item \fullcite{neuman14}.
\item \fullcite{gilardi22}.* (not suitable for response paper)
\end{itemize}

%\newpage
%\sloppy
%\renewcommand*{\bibfont}{\small}

%\setlength{\bibitemsep}{0.2em} % increase space between references
%\printbibliography

\end{document}

\bigskip

%\begin{center}
%Last updated: \today
%\end{center}




