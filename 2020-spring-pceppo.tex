\documentclass[abstract=on,parskip=full,headings=standardclasses,fontsize=11pt,paper=a4]{scrartcl}
\usepackage[paper=a4paper,left=21mm,right=21mm,top=25mm,bottom=25mm]{geometry}
\usepackage[utf8]{inputenc}
\usepackage[T1]{fontenc}
\usepackage[english]{babel}

\usepackage{adjustbox}
%\usepackage{amsmath}
\usepackage{graphicx}
%\usepackage{fullpage}
\usepackage{authblk}
\usepackage{setspace}
\usepackage{caption}
\usepackage{booktabs}
\usepackage{url}
\usepackage{comment}
\urlstyle{sf}
\usepackage{lmodern}
\usepackage[parfill]{parskip}
%\usepackage{url}
%\urlstyle{same}
\usepackage[small]{titlesec}
\usepackage{marvosym}

\setcounter{secnumdepth}{0}

\addto\captionsenglish{% Replace "english" with the language you use
  \renewcommand{\contentsname}%
    {Course Structure (Autumn Term 2019)}%
}



%\deffootnote[10pt]{10pt}{10pt}{\makebox[15pt][l]{\thefootnotemark\hspace{10pt}}}

% Use authoryear-comp to create: (Müller 2015, 2016) instead of (Müller 2015; Müller 2016)

% posscite function

\usepackage[style=authoryear-comp,
    maxcitenames=99,
    maxbibnames=99,
    doi=false,
    %sorting=ynt,
    firstinits=false,
    isbn=false,
    date=short,
    dashed=false,
    url=false,
    sortcites=false,
    backend=bibtex]{biblatex}

\makeatletter
\def\blx@maxline{77}
\makeatother


\DeclareNameFormat{labelname:poss}{% Based on labelname from biblatex.def
  \nameparts{#1}% Not needed if using Biblatex 3.4
  \ifcase\value{uniquename}%
    \usebibmacro{name:family}{\namepartfamily}{\namepartgiven}{\namepartprefix}{\namepartsuffix}%
  \or
    \ifuseprefix
      {\usebibmacro{name:first-last}{\namepartfamily}{\namepartgiveni}{\namepartprefix}{\namepartsuffixi}}
      {\usebibmacro{name:first-last}{\namepartfamily}{\namepartgiveni}{\namepartprefixi}{\namepartsuffixi}}%
  \or
    \usebibmacro{name:first-last}{\namepartfamily}{\namepartgiven}{\namepartprefix}{\namepartsuffix}%
  \fi
  \usebibmacro{name:andothers}%
  \ifnumequal{\value{listcount}}{\value{liststop}}{'s}{}}
\DeclareFieldFormat{shorthand:poss}{%
  \ifnameundef{labelname}{#1's}{#1}}
\DeclareFieldFormat{citetitle:poss}{\mkbibemph{#1}'s}
\DeclareFieldFormat{label:poss}{#1's}
\newrobustcmd*{\posscitealias}{%
  \AtNextCite{%
    \DeclareNameAlias{labelname}{labelname:poss}%
    \DeclareFieldAlias{shorthand}{shorthand:poss}%
    \DeclareFieldAlias{citetitle}{citetitle:poss}%
    \DeclareFieldAlias{label}{label:poss}}}
\newrobustcmd*{\posscite}{%
  \posscitealias%
  \textcite}
\newrobustcmd*{\Posscite}{\bibsentence\posscite}
\newrobustcmd*{\posscites}{%
  \posscitealias%
  \textcites}

\renewbibmacro{in:}{} % no "in" before article

\renewcommand*{\bibpagespunct}{\addcomma\space} % ":" instead of pp
\DeclareFieldFormat{pages}{#1}

% Colon after title
\renewcommand{\subtitlepunct}{\addcolon\addspace }

% Colon instead of pp in references
\renewcommand*{\bibpagespunct}{\addcolon\space}
\DeclareFieldFormat{pages}{#1}

% Colon after name in text
\renewcommand*{\postnotedelim}{\addcolon\space}
\DeclareFieldFormat{postnote}{#1}
\DeclareFieldFormat{multipostnote}{#1}

% Remove brackets around year in bibliography
\usepackage{xpatch,filecontents}

\xpatchbibmacro{date+extrayear}{%
  \printtext[parens]%
}{%
  \setunit*{\addperiod\space}%
  \printtext%
}{}{}

% Supresses URL accessed day
\AtEveryBibitem{%
  \ifentrytype{electronic}
    {}
    {\clearfield{urlyear}\clearfield{urlmonth}\clearfield{urlday}}}
%\DefineBibliographyStrings{english}{%
%urlseen = {Accessed},}

\renewbibmacro*{volume+number+eid}{% number of journal in brackets
 \printfield{volume}%
  %\setunit*{\adddot}% DELETED
  \setunit*{\addnbthinspace}% NEW (optional); there's also \addnbthinspace
  \printfield{number}%
  \setunit{\addcomma\space}%
  \printfield{eid}}
\DeclareFieldFormat[article]{number}{\mkbibparens{#1}}



% Change edition field

\DeclareFieldFormat{edition}%
                   {\ifinteger{#1}%
                    {\mkbibordedition{#1}\addthinspace{}edition}%
                    {#1\isdot}}

% New command to show doi, or url or isbn or issn field
% http://tex.stackexchange.com/questions/48400/biblatex-make-title-hyperlink-to-dois-url-or-isbn
\newbibmacro{string+doiurlisbn}[1]{%
  \iffieldundef{doi}{%
    \iffieldundef{url}{%
      \iffieldundef{isbn}{%
        \iffieldundef{issn}{%
          #1%
        }{%
          \href{http://books.google.com/books?vid=ISSN\thefield{issn}}{#1}%
        }%
      }{%
        \href{http://books.google.com/books?vid=ISBN\thefield{isbn}}{#1}%
      }%
    }{%
      \href{\thefield{url}}{#1}%
    }%
  }{%
    \href{http://dx.doi.org/\thefield{doi}}{#1}%
  }%
}

% Necessary to remove dot after question mark in title
%\newcommand{\killpunct}[1]{}    

% Make full stop after title and before quotation marks in title field
\DeclareFieldFormat{title}{\usebibmacro{string+doiurlisbn}{\mkbibemph{#1}}}
\DeclareFieldFormat[article,incollection,unpublished,phdthesis]{title}%
    {\usebibmacro{string+doiurlisbn}{\mkbibquote{#1}}}
   % {\usebibmacro{string+doiurlisbn}{\mkbibquote{#1.\isdot}}}

\renewcommand*{\newunitpunct}{.\space}


\bibliography{/Users/smueller/Documents/GitHub/literature/muellerlibrary.bib}
%\bibliography{/Users/stefan/GitHub/literature/muellerlibrary.bib}


\usepackage{xcolor}
\definecolor{JournalBlue}{RGB}{0, 12, 146}
%https://en.wikibooks.org/wiki/LaTeX/Colors
\usepackage[colorlinks=true, linkcolor=JournalBlue, filecolor=black, urlcolor=JournalBlue, pdfborder={0 0 0},citecolor=JournalBlue]{hyperref}%RoyalBlue
%\usepackage[colorlinks]{hyperref}

\clubpenalty = 10000 
\widowpenalty = 10000 
\displaywidowpenalty = 10000

\setlength\parindent{0pt}


\usepackage{titlesec}
\titleformat{\section}
   {\normalfont\large\bfseries}{\thesection}{1em}{}

   

\begin{document}
	
\singlespacing

\noindent
\adjustbox{valign=t}{\begin{minipage}{0.38\textwidth}% adapt widths of minipages to your needs
\includegraphics[width=\linewidth]{pictures/uzh_logo_en}
\end{minipage}}%
\hfill%
\adjustbox{valign=t}{\begin{minipage}{0.62\textwidth}\raggedleft
{%\footnotesize
\textbf{Stefan Müller, PhD} \\
Senior Researcher\\
%Chair of Policy Analysis \\
Department of Political Science \\
University of Zurich \\
\Letter\ \href{mailto:mueller@ipz.uzh.ch}{\textsf{mueller@ipz.uzh.ch}} \\
\url{https://muellerstefan.net} \\
}
\end{minipage}}

\singlespacing
\vspace{1cm}

\begin{center}
{\large Spezialisierung \href{https://studentservices.uzh.ch/uzh/anonym/vvz/index.html#/details/2019/004/E/50981122}{615a006a}: Spring Term 2020} \\ 
\bigskip

{\Large \textbf{Promises Made, Promises Kept?\\ \vspace{3mm} Party Competition, Election Pledges, and Policy Outcomes}} 
\bigskip

%{\large  \textcolor{red}{Draft (last update: \today)}}

{\large  {Last update: \today}}\\
\bigskip

%Latest version: \url{https://muellerstefan.net/teaching/2019-autumn-pceppo.pdf}
\end{center}

\vspace{1.5cm}



\hrule
\medskip
% first column
\begin{minipage}[t]{0.5\textwidth}
Term: Spring term 2020 \\
Time: 28/02; 20/03; 08/05; 08:00--12:00 \\
Room:  \href{https://www.plaene.uzh.ch/AFL}{AFL-E-020} (Affolternstr. 56) \\
ECTS: 15
\end{minipage}
%second column
\begin{minipage}[t]{0.49\textwidth}
\begin{flushright}
Lecturer: Stefan Müller \\
E-mail: \href{mailto:mueller@ipz.uzh.ch}{\textsf{mueller@ipz.uzh.ch}} \\
Office:  AFL-E-003 (Affolternstr. 56) \\
Office hours: email for appointment \\
\end{flushright}
\end{minipage}
\medskip
\vspace{2.5mm}
\hrule 

\section*{Course Content}

Do parties keep their promises or are politicians ``pledge breakers''? Are promises in certain policy areas more likely to be fulfilled? In what policy areas do parties differ in terms of their positions and issue emphasis? And how can we measure election promises and latent party positions reliably? In this seminar, we will first compare theories of policy-making and connect them with theories of party competition. Second, we discuss different approaches of measuring party positions, political ideology, and the saliency of policy areas in detail. Third, we will analyse in detail how party competition influences policy-making and identify the circumstances under which parties adjust their positions.

The second semester includes an applied introduction to quantitative text analysis in order to classify text into policy areas and measure party positions. The aim of the seminar is the development of an innovative research design that forms the basis for a BA thesis.


\section*{Details}

\begin{itemize}
\item BA ``Spezialisierung''
\item  Language: German
\item Grading: Bachelor thesis (100\%)
\end{itemize}

\section*{Learning Outcomes}

\begin{enumerate}
\item Extensive knowledge of central theories of representation, the mandate model of democracy, and party competition. 
\item Detailed insights into past and current approaches to study questions about pledge fulfilment, party positions, responsiveness and issue ownership 
\item Critical reading and discussing  complex academic literature and diverse methodological approaches
\item Planning and writing a research design which forms the basis of the  BA thesis, to be written in the second part of the module (FS 2020)
\end{enumerate}

\section*{Introductory Readings}

The seminar does not build on a single text book, but relies mostly on papers and chapters of books. For  a general overview of the course content, I recommend the following books:

\begin{itemize}
\item \fullcite{powell00}.
\item \fullcite{dalton11}.
\item \fullcite{gallagher11}.
\item \fullcite{volkens13}.
\end{itemize}


\section*{Technical Background and Prerequisites}

The course requires good knowledge of general approaches and theories of political science and basic prior knowledge with research design and quantitative methods. The following books provide very good introductions to research design and applied quantitative methods.

\subsection*{Research Design and Quantitative Methods}
\begin{itemize}
\item \fullcite{king94}.
\item \fullcite{gerring01}.
\item \fullcite{kellstedt19}.
\item \fullcite{imai17}.
\item \fullcite{wickham17}.
\end{itemize}

\subsection*{Academic Writing}
\begin{itemize}
\item \fullcite{heard16}.
\end{itemize}


\section*{Syllabus Modification Rights}

I reserve the right to reasonably alter the elements of the syllabus at any time by adjusting the reading list to keep pace with the course schedule. Moreover, I may change the content of specific sessions depending on the participants' prior knowledge and research interests.


\section*{Expectations and Grading}

Students will write a Bachelor thesis which will count towards 100\% of this module. The \href{https://www.ipz.uzh.ch/dam/jcr:0003a1a3-4055-43ca-b2ad-2579218e06f4/IPZ_Richtlinien_BA_Arbeit.pdf}{expectations regarding the BA thesis are provided at the department's website}
Students must follow the \href{https://www.ipz.uzh.ch/dam/jcr:ffffffff-f62d-eb71-0000-000002d0c74d/Merkblatt_ZitierenBibliographieren_vult_20080609.pdf}{guidelines for citing academic references}.


\section{Week 1: Organisation and Introduction (18.09.2019)}

\begin{itemize}
\renewcommand\labelitemi{--}
\item Expectations
\item Discussion of syllabus
\item Initial information on presentations, the research proposal, and the second term 
\end{itemize}


\subsubsection*{Mandatory Readings}
\begin{itemize}
\item \fullcite[ch. 1]{clarke18}.
\end{itemize}


\section{Week 2: Parties and Party Systems (25.09.2019)}

\begin{itemize}
\renewcommand\labelitemi{--}
\item  What are political parties?
\item What does Lijphart mean by the Westminter Model of Democracy and the Consensus Model of Democracy?
\item How can we distinguish between different types of democracies?
\end{itemize}

\subsubsection*{Mandatory Readings}
\begin{itemize}
%\item \fullcite{boix07}.
%\item \fullcite{katz09}.
\item \fullcite[ch. 1--3]{lijphart12}.
%\item \fullcite[Kapitel 1]{dalton11}.
%\item \fullcite[ch. 1--2]{powell00}.
\end{itemize}



\section{Week 3:  Mandate Model of Democracy (02.10.2019)}



\begin{itemize}
\renewcommand\labelitemi{--}
\item What is the `democratic mandate'? 
\item How we measure campaign promises/pledges?
\item Do parties fulfil their promises?
\end{itemize}

\subsubsection*{Mandatory Readings}
\begin{itemize}
\item \fullcite[29--40]{manin99}.
\item \fullcite{thomson17}.
\item \fullcite{thomson18}.
\end{itemize}



\end{document}


