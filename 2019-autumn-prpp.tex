\documentclass[abstract=on,parskip=full,headings=standardclasses,fontsize=11pt,paper=a4]{scrartcl}
\usepackage[paper=a4paper,left=21mm,right=21mm,top=25mm,bottom=25mm]{geometry}
\usepackage[utf8]{inputenc}
\usepackage[T1]{fontenc}
\usepackage[english]{babel}

\usepackage{adjustbox}
%\usepackage{amsmath}
\usepackage{graphicx}
%\usepackage{fullpage}
\usepackage{authblk}
\usepackage{setspace}
\usepackage{caption}
\usepackage{booktabs}
\usepackage{url}
\usepackage{comment}
\urlstyle{sf}
\usepackage{lmodern}
\usepackage[parfill]{parskip}
%\usepackage{url}
%\urlstyle{same}
\usepackage[small]{titlesec}
\usepackage{marvosym}

\setcounter{secnumdepth}{0}

\addto\captionsenglish{% Replace "english" with the language you use
  \renewcommand{\contentsname}%
    {Course Structure (Autumn Term 2019)}%
}


%\deffootnote[10pt]{10pt}{10pt}{\makebox[15pt][l]{\thefootnotemark\hspace{10pt}}}

% Use authoryear-comp to create: (Müller 2015, 2016) instead of (Müller 2015; Müller 2016)

% posscite function

\usepackage[style=authoryear-comp,
    maxcitenames=99,
    maxbibnames=99,
    doi=false,
    %sorting=ynt,
    firstinits=false,
    isbn=false,
    date=short,
    dashed=false,
    url=false,
    sortcites=false,
    backend=bibtex]{biblatex}

\makeatletter
\def\blx@maxline{77}
\makeatother


\DeclareNameFormat{labelname:poss}{% Based on labelname from biblatex.def
  \nameparts{#1}% Not needed if using Biblatex 3.4
  \ifcase\value{uniquename}%
    \usebibmacro{name:family}{\namepartfamily}{\namepartgiven}{\namepartprefix}{\namepartsuffix}%
  \or
    \ifuseprefix
      {\usebibmacro{name:first-last}{\namepartfamily}{\namepartgiveni}{\namepartprefix}{\namepartsuffixi}}
      {\usebibmacro{name:first-last}{\namepartfamily}{\namepartgiveni}{\namepartprefixi}{\namepartsuffixi}}%
  \or
    \usebibmacro{name:first-last}{\namepartfamily}{\namepartgiven}{\namepartprefix}{\namepartsuffix}%
  \fi
  \usebibmacro{name:andothers}%
  \ifnumequal{\value{listcount}}{\value{liststop}}{'s}{}}
\DeclareFieldFormat{shorthand:poss}{%
  \ifnameundef{labelname}{#1's}{#1}}
\DeclareFieldFormat{citetitle:poss}{\mkbibemph{#1}'s}
\DeclareFieldFormat{label:poss}{#1's}
\newrobustcmd*{\posscitealias}{%
  \AtNextCite{%
    \DeclareNameAlias{labelname}{labelname:poss}%
    \DeclareFieldAlias{shorthand}{shorthand:poss}%
    \DeclareFieldAlias{citetitle}{citetitle:poss}%
    \DeclareFieldAlias{label}{label:poss}}}
\newrobustcmd*{\posscite}{%
  \posscitealias%
  \textcite}
\newrobustcmd*{\Posscite}{\bibsentence\posscite}
\newrobustcmd*{\posscites}{%
  \posscitealias%
  \textcites}

\renewbibmacro{in:}{} % no "in" before article

\renewcommand*{\bibpagespunct}{\addcomma\space} % ":" instead of pp
\DeclareFieldFormat{pages}{#1}

% Colon after title
\renewcommand{\subtitlepunct}{\addcolon\addspace }

% Colon instead of pp in references
\renewcommand*{\bibpagespunct}{\addcolon\space}
\DeclareFieldFormat{pages}{#1}

% Colon after name in text
\renewcommand*{\postnotedelim}{\addcolon\space}
\DeclareFieldFormat{postnote}{#1}
\DeclareFieldFormat{multipostnote}{#1}

% Remove brackets around year in bibliography
\usepackage{xpatch,filecontents}

\xpatchbibmacro{date+extrayear}{%
  \printtext[parens]%
}{%
  \setunit*{\addperiod\space}%
  \printtext%
}{}{}

% Supresses URL accessed day
\AtEveryBibitem{%
  \ifentrytype{electronic}
    {}
    {\clearfield{urlyear}\clearfield{urlmonth}\clearfield{urlday}}}
%\DefineBibliographyStrings{english}{%
%urlseen = {Accessed},}

\renewbibmacro*{volume+number+eid}{% number of journal in brackets
 \printfield{volume}%
  %\setunit*{\adddot}% DELETED
  \setunit*{\addnbthinspace}% NEW (optional); there's also \addnbthinspace
  \printfield{number}%
  \setunit{\addcomma\space}%
  \printfield{eid}}
\DeclareFieldFormat[article]{number}{\mkbibparens{#1}}



% Change edition field

\DeclareFieldFormat{edition}%
                   {\ifinteger{#1}%
                    {\mkbibordedition{#1}\addthinspace{}edition}%
                    {#1\isdot}}

% New command to show doi, or url or isbn or issn field
% http://tex.stackexchange.com/questions/48400/biblatex-make-title-hyperlink-to-dois-url-or-isbn
\newbibmacro{string+doiurlisbn}[1]{%
  \iffieldundef{doi}{%
    \iffieldundef{url}{%
      \iffieldundef{isbn}{%
        \iffieldundef{issn}{%
          #1%
        }{%
          \href{http://books.google.com/books?vid=ISSN\thefield{issn}}{#1}%
        }%
      }{%
        \href{http://books.google.com/books?vid=ISBN\thefield{isbn}}{#1}%
      }%
    }{%
      \href{\thefield{url}}{#1}%
    }%
  }{%
    \href{http://dx.doi.org/\thefield{doi}}{#1}%
  }%
}

% Necessary to remove dot after question mark in title
%\newcommand{\killpunct}[1]{}    

% Make full stop after title and before quotation marks in title field
\DeclareFieldFormat{title}{\usebibmacro{string+doiurlisbn}{\mkbibemph{#1}}}
\DeclareFieldFormat[article,incollection,unpublished,phdthesis]{title}%
    {\usebibmacro{string+doiurlisbn}{\mkbibquote{#1}}}
   % {\usebibmacro{string+doiurlisbn}{\mkbibquote{#1.\isdot}}}

\renewcommand*{\newunitpunct}{.\space}


\bibliography{/Users/smueller/Documents/GitHub/literature/muellerlibrary.bib}
%\bibliography{/Users/stefan/GitHub/literature/muellerlibrary.bib}


\usepackage{xcolor}
\definecolor{JournalBlue}{RGB}{0, 12, 146}
%https://en.wikibooks.org/wiki/LaTeX/Colors
\usepackage[colorlinks=true, linkcolor=JournalBlue, filecolor=black, urlcolor=JournalBlue, pdfborder={0 0 0},citecolor=JournalBlue]{hyperref}%RoyalBlue
%\usepackage[colorlinks]{hyperref}

\clubpenalty = 10000 
\widowpenalty = 10000 
\displaywidowpenalty = 10000

\setlength\parindent{0pt}


\usepackage{titlesec}
\titleformat{\section}
   {\normalfont\large\bfseries}{\thesection}{1em}{}

   

\begin{document}
	
\singlespacing

\noindent
\adjustbox{valign=t}{\begin{minipage}{0.38\textwidth}% adapt widths of minipages to your needs
\includegraphics[width=\linewidth]{pictures/uzh_logo_en}
\end{minipage}}%
\hfill%
\adjustbox{valign=t}{\begin{minipage}{0.62\textwidth}\raggedleft
{%\footnotesize
\textbf{Stefan Müller, PhD} \\
Postdoctoral Researcher\\
%Chair of Policy Analysis \\
Department of Political Science \\
University of Zurich \\
\Letter\ \href{mailto:mueller@ipz.uzh.ch}{\textsf{mueller@ipz.uzh.ch}} \\
\url{https://muellerstefan.net} \\
}
\end{minipage}}

\singlespacing
\vspace{1cm}

\begin{center}
{\large Pre-Research Seminar: \href{https://studentservices.uzh.ch/uzh/anonym/vvz/index.html?userType=QS#/details/2019/003/E/50990175}{615-501b}
Autumn Term 2019} \\ 
\bigskip

{\Large \textbf{Political Representation and Policy Preferences}} 
\bigskip

%{\large  \textcolor{red}{Draft (last update: \today)}}

{\large  {Last update: \today}}\\
\bigskip

%Latest version: \url{https://muellerstefan.net/teaching/2019-autumn-pceppo.pdf}
\end{center}

\vspace{1.5cm}



\hrule
\medskip
% first column
\begin{minipage}[t]{0.5\textwidth}
Term: Autumn term 2019 \\
Time: Tuesday, 16:15--18:00 \\
Room:  \href{https://www.plaene.uzh.ch/AFL}{AFL-F-172/173} (Affolternstr. 56) \\
ECTS: 6
\end{minipage}
%second column
\begin{minipage}[t]{0.49\textwidth}
\begin{flushright}
Lecturer: Stefan Müller \\
E-mail: \href{mailto:mueller@ipz.uzh.ch}{\textsf{mueller@ipz.uzh.ch}} \\
Office:  AFL-H-349 (Affolternstr. 56) \\
Office hours: email for appointment \\
\end{flushright}
\end{minipage}
\medskip
\vspace{2.5mm}
\hrule 

\section*{Course Content}

When do political parties fulfil or break election promises? How can researchers and citizens identify political promises? How does pledge fulfilment relate to theories of political representation? And why do parties and politicians change their positions? 
These questions will be discussed in this pre-research seminar. First, we revisit classic theories of representation and policy-making. Afterwards, we turn to the definition and measurement of public opinion, different styles of representation, and the concepts of responsiveness and congruence. Based on  these theoretical foundations, we analyse party competition as well as salience and latent policy positions.  We  also  discuss and apply to text-as-data methods, and revisit some of the most important aspects for designing a research project. This pre-research seminar is aimed at students who would like to attend the seminar together with the research seminar in the spring term 2020 as a one-year course.

%The second semester includes an applied introduction to quantitative text analysis in order to classify text into policy areas and measure party positions. The aim of the seminar is the development of an innovative research design that forms the basis for a BA thesis.


\section*{Details}

\begin{itemize}
\item Pre-Research Seminar (Autumn term 2019); Research seminar (Spring term 2020)
\item  Language: English
\item Grading: Weekly wiki posts about course literature: 90\%;  Outline of research proposal: 10\%
\end{itemize}

\section*{Learning Outcomes}

\begin{enumerate}
\item Extensive knowledge of central theories of representation, the mandate model of democracy, and party competition. 
\item Detailed insights into past and current approaches to study questions about pledge fulfilment, party positions, responsiveness and issue ownership.
\item Critical reading and discussing  complex academic literature and diverse methodological approaches.
\item Planning and writing a research design which forms the basis of the the empirical research paper (Forschungsarbeit), to be written in the second part of the module (FS 2020).
\end{enumerate}

\section*{Introductory Readings}

The seminar does not build on a single text book, but relies mostly on papers and chapters of books. For  a general overview of the course content, I recommend the following books:

\begin{itemize}
\item \fullcite{powell00}.
\item \fullcite{dalton11}.
\item \fullcite{gallagher11}.
\item \fullcite{volkens13}.
\end{itemize}


\section*{Technical Background and Prerequisites}

The course requires good knowledge of general approaches and theories of political science and basic prior knowledge with research design and quantitative methods. The following books provide very good introductions to empirical research designs and applied quantitative methods.

\subsection*{Research Design and Quantitative Methods}
\begin{itemize}
\item \fullcite{king94}.
\item \fullcite{gerring01}.
%\item \fullcite{kellstedt19}.
\item \fullcite{imai17}.
\item \fullcite{wickham17}.
\item \fullcite{harrell}.
\end{itemize}

\subsection*{Academic Writing}
\begin{itemize}
\item \fullcite{heard16}.
\end{itemize}


\section*{Syllabus Modification Rights}

I reserve the right to reasonably alter the elements of the syllabus at any time by adjusting the reading list to keep pace with the course schedule. Moreover, I may change the content of specific sessions depending on the participants' prior knowledge and research interests.


\section*{Expectations and Grading}


\begin{itemize}
\item Students  must  read all papers or chapters assigned under \textbf{Mandatory Readings}.  I also add optional readings which can be used as additional evidence for the weekly wiki posts or serve as a preparation for the empirical research paper.

\item Students will upload  weekly wiki posts  at OLAT. These posts are comparable to response papers. Each post (between 500 and 750 words) should \textit{critically} discuss the required readings for the respective session. The post must be submitted  at last 3 (!) hours before the start of the seminar session. Students should \textit{not}  merely summarise the readings, but  discuss weaknesses -- either by comparing the papers critically  or by making suggestions on how to improve the theory, data, or methods.  The posts are supposed to encourage students to think critically about the readings.  Students must submit at least 9 posts with passable quality, but have two `jokers': students can submit up to 11 posts and the 2 posts with the lowest grades will not count towards the final grade. More information on the posts will be provided during the course. The nine posts can be written in English or German and  count towards 90\% of the grade. 

\item Students will submit an outline of the empirical research paper. This outline counts towards 10\% of the final grade and will present a testable research question,  theoretical expectations,  the dependent variable, a preliminary overview of the data to be collected or analysed for the Forschungsarbeit, and a description of the methodological approach. The outline can be written in English or German, and must be submitted before \textbf{December 13, 2019 (8:00pm CET)}. Concrete information on the length of the outline will be provided in class.  In spring term 2020, students will use submit a  more concrete research design and the final research paper. Additional information on the research design and research paper will be provided at the beginning of spring term 2020.


%\item Students will prepare a  \textbf{Presentation} of one of the optional readings. This presentation counts 40\% towards the grade for this term. Dates and texts for presentations will be assigned in the third week of the seminar. The presentation includes a brief and concise discussion of the paper or book, with particular reference to the puzzle, research question, hypotheses, and results. The main part of the presentation should be devoted to a critical assessment of the paper: what open questions remain and how has subsequent research addressed these questions? What are weaknesses of the methods or case selection strategy? Are results internally and externally valid and generalisable? And how would you improve or extend the study?

%\item Students also submit a \textbf{Research Proposal} which counts towards 60\% of the final grade. The research proposal must not exceed 4,000 words  (including bibliography, captions, and footnotes).  The proposal  should identify an open research question, a discussion of the variation to be explained, and the importance of the research question. Moreover, the students should specify observable implications, the measurement and conceptualisation of the dependent and main independent variable, and propose a methodological approach to analyse this question. More details on these aspects and the research design will be provided throughout the seminar. The research design must be submitted via \href{https://lms.uzh.ch}{OLAT} as a \texttt{PDF} document before \textbf{December 13, 2019 (8:00pm CET)}. 
\end{itemize}



\newpage

\tableofcontents

\section{Week 1: Organisation and Introduction (September 17, 2019)}

\begin{itemize}
\renewcommand\labelitemi{--}
\item Expectations
\item Discussion of syllabus
\item Initial information on wiki posts, the outline of the  research proposal, and the second term 
\end{itemize}


\section{Week 2: Parties and Party Systems (September 24, 2019)}

\begin{itemize}
\renewcommand\labelitemi{--}
\item  What are political parties?
\item How have political parties evolved over time?
\item What constitutes a party system?
\end{itemize}

\subsubsection*{Mandatory Readings}
\begin{itemize}
\item \fullcite[ch. 1]{clarke18}.
%\item \fullcite{boix07}.
%\item \fullcite{katz09}.
\item \fullcite[ch. 1--3]{lijphart12}.
%\item \fullcite[Kapitel 1]{dalton11}.
%\item \fullcite[ch. 1--2]{powell00}.
\end{itemize}



\section{Week 3:  Mandate Model of Democracy (October 1, 2019)}



\begin{itemize}
\renewcommand\labelitemi{--}
\item What is the `democratic mandate'? 
\item How we measure campaign promises/pledges?
\item Do parties fulfil their promises?
\end{itemize}

\subsubsection*{Mandatory Readings}
\begin{itemize}
\item \fullcite[29--40]{manin99}.
\item \fullcite{thomson17}.
\item \fullcite{thomson18}.
\end{itemize}


\section{Week 4: Measuring Public Opinion (October 7, 2019)}


\begin{itemize}
\renewcommand\labelitemi{--}
\item What is public opinion?
\item How can we measure public opinion? 
\item What are advantages and shortcomings of different survey instruments?
\end{itemize}


\subsubsection*{Mandatory Readings}
\begin{itemize}
\item \fullcite{squire88}.
\item \fullcite{berinsky17}.
\end{itemize}


\subsubsection*{Optional}
\begin{itemize}
\item \fullcite{chong07b}.
\item \fullcite{boynton04}.
\end{itemize}


\section{Week 5: Politicians: Trustees or Delegates? (October 15, 2019)}


\begin{itemize}
\renewcommand\labelitemi{--}
\item What roles do politicians take during campaigns and in office? 
\item What are differences between the trustee and delegate model of representation? What type of representation is preferable?
\end{itemize}


\subsubsection*{Mandatory Readings}
\begin{itemize}
\item \fullcite{mueller06}.
\item \fullcite{bowler17}.
\end{itemize}


\subsubsection*{Optional}
\begin{itemize}
\item \fullcite{mueller00b}.
\item \fullcite{oennudottir16}.
%\item \fullcite{werner18}.
\end{itemize}



\section{Week 6: Economic Voting and the Cost of Governing (October 22, 2019)}


\begin{itemize}
\renewcommand\labelitemi{--}
\item What is democratic accountability?
\item Why do government parties  regularly lose public support at the next election?
\end{itemize}

\subsubsection*{Mandatory Readings}

\begin{itemize}
\item \fullcite[ch. 5]{achen16}.
\item \fullcite{kluever19}.
\end{itemize}


\subsubsection*{Optional}
\begin{itemize}
\item \fullcite{healy13}.
\item \fullcite{reif80}.
\item \fullcite{fowler18}.
\item \fullcite{sances17}.
%\item \fullcite{muellerlouwerse}.
\end{itemize}

 

\section{Week 7:  Responsiveness (October 29, 2019)}



\begin{itemize}
\renewcommand\labelitemi{--}
\item What are the differences between accountability and responsiveness?
\item Do parties and politicians react to public opinion? 
\end{itemize}

\subsubsection*{Mandatory Readings}
\begin{itemize}
\item \fullcite{wlezien95}.
\item \fullcite{kluever16}.
\end{itemize}

\subsubsection*{Optional}
\begin{itemize}
%\item \fullcite{eulau77}.
\item \fullcite{powell04b}.
\item \fullcite{page83}.
\item \fullcite{stimson95}.
%\item \fullcite{soroka10}.
\end{itemize}


\section{Week 8: Party Competition (November 5, 2019)}

%\begin{comment}
\begin{itemize}
\renewcommand\labelitemi{--}
\item What goals do parties and politicians pursue?
\item How do parties compete with each other, and how can we measure party competition?
\end{itemize}

\subsubsection*{Mandatory Readings}
\begin{itemize}
\item \fullcite{strom90}.
\item \fullcite{somertopcu15}.
%\item \fullcite{adams09b}. % BJPS
\end{itemize}

\subsubsection*{Optional}
\begin{itemize}
\item \fullcite{stokes63}.
\item \fullcite{greenpedersen07}.
\item \fullcite{tavits07}.
\item \fullcite{boehmelt16}.
%\item \fullcite{wagner14}.
%\item \fullcite{spoon15}.
\end{itemize}
%\end{comment}


\section{Week 9:  Party Positions, Salience and Issue Ownership (November 12, 2019)}


\begin{itemize}
\renewcommand\labelitemi{--}
\item What are differences between positions, salience, and issue ownership?
\item How can we measure latent policy positions? 
\item What are methodological difficulties when measuring party positions?
\end{itemize}

\subsubsection*{Mandatory Readings}
\begin{itemize}
\item \fullcite{laver14}.
\item \fullcite{budge15}.
\end{itemize}


\subsubsection*{Optional}
\begin{itemize}
\item \fullcite{mikhaylov12}.
\item \fullcite{somertopcu15}.
\item \fullcite{bischof19}.
\end{itemize}


\section{Week 10:  Measuring Party Positions (November 19, 2019)}

\begin{itemize}
\renewcommand\labelitemi{--}
\item Which datasets are available to measure party positions and issue salience?
\item What software tools and methods can be used to derive latent party positions and classify issue salience?
\end{itemize}


\subsubsection*{Mandatory Readings}
\begin{itemize}
\item \fullcite{laver03}.
\item \fullcite{budge13}.
\item \fullcite{benoit16}.
\end{itemize}


\subsubsection*{Optional}
\begin{itemize}
\item \fullcite{slapin08}.
\item \fullcite{merz16}.
\end{itemize}


\section{Week 11:  Application: Party Positions and Issue Salience (November 26, 2019)}

\begin{itemize}
\renewcommand\labelitemi{--}
\item How can we apply the methods discussed in week 10 to textual data using the \texttt{quanteda} \textsf{R} package?
\end{itemize}

\subsubsection*{Mandatory Readings}
\begin{itemize}
\item \fullcite{grimmer13}.
\item \fullcite{benoit18}.
\item \fullcite{welbers17}.
\end{itemize}


\subsubsection*{Optional}
\begin{itemize}
\item \fullcite{watanabemueller}.
\end{itemize}





\section{Week 12: Representation in the Age of Digital Democracy (December 3, 2019)}


\begin{itemize}
\renewcommand\labelitemi{--}
\item How does the internet change democratic decision making and representation?
\item Do politicians and parties react to online discussions?
\end{itemize}

\subsubsection*{Mandatory Readings}
\begin{itemize}
\item \fullcite{king17b}.
\item \fullcite{guess19}.
\end{itemize}


\subsubsection*{Optional}
\begin{itemize}
\item \fullcite{farrell12}.
\item \fullcite{barbera19}.
\item \fullcite{neuman14}.
\end{itemize}


\section{Week 13:  Research Design: Research Question and Dependent Variable  (October 10, 2019)}

\begin{itemize}
\renewcommand\labelitemi{--}
\item \fullcite[ch. 1]{firebaugh08}.
\item \fullcite[ch.1; 107--12]{king94}.
\end{itemize}

\subsubsection*{Optional}
\begin{itemize}
\item \fullcite{adcock01}.
\end{itemize}

\section{Week 14: Research Design: Falsifiability and Causal Inference (December 17, 2019)}


\subsubsection*{Mandatory Readings}
\begin{itemize}
\item \fullcite[ch. 3]{king94}.
\end{itemize}


\subsubsection*{Optional}
\begin{itemize}
\item \fullcite{holland86}.
\item \fullcite[ch. 7]{gerring01}.
\end{itemize}



%\section{Week 14: Feedback on Research Proposal (18.12.2019)}


%In the last session of the term, we will discuss the outlines of the research proposals which will form the basis for the empirical research paper in spring term 2020. Moreover, we will discuss the outline of the second part of the course in spring term 2020.


%\newpage
\sloppy
\renewcommand*{\bibfont}{\small}

\setlength{\bibitemsep}{0.2em} % increase space between references
%\printbibliography

\bigskip

%\begin{center}
%Last updated: \today
%\end{center}

\end{document}


