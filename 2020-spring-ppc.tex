\documentclass[abstract=on,parskip=full,headings=standardclasses,fontsize=11pt,paper=a4]{scrartcl}
%\usepackage[paper=a4paper,left=21mm,right=21mm,top=25mm,bottom=25mm]{geometry}
\usepackage[paper=a4paper,left=20mm,right=15mm,top=25mm,bottom=25mm]{geometry}

\usepackage[utf8]{inputenc}
\usepackage[T1]{fontenc}
\usepackage[english]{babel}

\usepackage{adjustbox}
%\usepackage{amsmath}
\usepackage{graphicx}
%\usepackage{fullpage}
\usepackage{authblk}
\usepackage{setspace}
\usepackage{caption}
\usepackage{booktabs}
\usepackage{url}
\usepackage{comment}
\urlstyle{sf}
\usepackage{lmodern}
\usepackage[parfill]{parskip}
%\usepackage{url}
%\urlstyle{same}
\usepackage[small]{titlesec}
\usepackage{marvosym}

\setcounter{secnumdepth}{0}

\addto\captionsenglish{% Replace "english" with the language you use
  \renewcommand{\contentsname}%
    {Course Structure}%
}



%\deffootnote[10pt]{10pt}{10pt}{\makebox[15pt][l]{\thefootnotemark\hspace{10pt}}}

% Use authoryear-comp to create: (Müller 2015, 2016) instead of (Müller 2015; Müller 2016)

% posscite function

\usepackage[style=authoryear-comp,
    maxcitenames=99,
    maxbibnames=99,
    doi=false,
    %sorting=ynt,
    firstinits=false,
    isbn=false,
    date=short,
    dashed=false,
    url=false,
    sortcites=false,
    backend=bibtex]{biblatex}

\makeatletter
\def\blx@maxline{77}
\makeatother


\DeclareNameFormat{labelname:poss}{% Based on labelname from biblatex.def
  \nameparts{#1}% Not needed if using Biblatex 3.4
  \ifcase\value{uniquename}%
    \usebibmacro{name:family}{\namepartfamily}{\namepartgiven}{\namepartprefix}{\namepartsuffix}%
  \or
    \ifuseprefix
      {\usebibmacro{name:first-last}{\namepartfamily}{\namepartgiveni}{\namepartprefix}{\namepartsuffixi}}
      {\usebibmacro{name:first-last}{\namepartfamily}{\namepartgiveni}{\namepartprefixi}{\namepartsuffixi}}%
  \or
    \usebibmacro{name:first-last}{\namepartfamily}{\namepartgiven}{\namepartprefix}{\namepartsuffix}%
  \fi
  \usebibmacro{name:andothers}%
  \ifnumequal{\value{listcount}}{\value{liststop}}{'s}{}}
\DeclareFieldFormat{shorthand:poss}{%
  \ifnameundef{labelname}{#1's}{#1}}
\DeclareFieldFormat{citetitle:poss}{\mkbibemph{#1}'s}
\DeclareFieldFormat{label:poss}{#1's}
\newrobustcmd*{\posscitealias}{%
  \AtNextCite{%
    \DeclareNameAlias{labelname}{labelname:poss}%
    \DeclareFieldAlias{shorthand}{shorthand:poss}%
    \DeclareFieldAlias{citetitle}{citetitle:poss}%
    \DeclareFieldAlias{label}{label:poss}}}
\newrobustcmd*{\posscite}{%
  \posscitealias%
  \textcite}
\newrobustcmd*{\Posscite}{\bibsentence\posscite}
\newrobustcmd*{\posscites}{%
  \posscitealias%
  \textcites}

\renewbibmacro{in:}{} % no "in" before article

\renewcommand*{\bibpagespunct}{\addcomma\space} % ":" instead of pp
\DeclareFieldFormat{pages}{#1}

% Colon after title
\renewcommand{\subtitlepunct}{\addcolon\addspace }

% Colon instead of pp in references
\renewcommand*{\bibpagespunct}{\addcolon\space}
\DeclareFieldFormat{pages}{#1}

% Colon after name in text
\renewcommand*{\postnotedelim}{\addcolon\space}
\DeclareFieldFormat{postnote}{#1}
\DeclareFieldFormat{multipostnote}{#1}

% Remove brackets around year in bibliography
\usepackage{xpatch,filecontents}

\xpatchbibmacro{date+extrayear}{%
  \printtext[parens]%
}{%
  \setunit*{\addperiod\space}%
  \printtext%
}{}{}

% Supresses URL accessed day
\AtEveryBibitem{%
  \ifentrytype{electronic}
    {}
    {\clearfield{urlyear}\clearfield{urlmonth}\clearfield{urlday}}}
%\DefineBibliographyStrings{english}{%
%urlseen = {Accessed},}

\renewbibmacro*{volume+number+eid}{% number of journal in brackets
 \printfield{volume}%
  %\setunit*{\adddot}% DELETED
  \setunit*{\addnbthinspace}% NEW (optional); there's also \addnbthinspace
  \printfield{number}%
  \setunit{\addcomma\space}%
  \printfield{eid}}
\DeclareFieldFormat[article]{number}{\mkbibparens{#1}}



% Change edition field

\DeclareFieldFormat{edition}%
                   {\ifinteger{#1}%
                    {\mkbibordedition{#1}\addthinspace{}edition}%
                    {#1\isdot}}

% New command to show doi, or url or isbn or issn field
% http://tex.stackexchange.com/questions/48400/biblatex-make-title-hyperlink-to-dois-url-or-isbn
\newbibmacro{string+doiurlisbn}[1]{%
  \iffieldundef{doi}{%
    \iffieldundef{url}{%
      \iffieldundef{isbn}{%
        \iffieldundef{issn}{%
          #1%
        }{%
          \href{http://books.google.com/books?vid=ISSN\thefield{issn}}{#1}%
        }%
      }{%
        \href{http://books.google.com/books?vid=ISBN\thefield{isbn}}{#1}%
      }%
    }{%
      \href{\thefield{url}}{#1}%
    }%
  }{%
    \href{http://dx.doi.org/\thefield{doi}}{#1}%
  }%
}

% Necessary to remove dot after question mark in title
%\newcommand{\killpunct}[1]{}    

% Make full stop after title and before quotation marks in title field
\DeclareFieldFormat{title}{\usebibmacro{string+doiurlisbn}{\mkbibemph{#1}}}
\DeclareFieldFormat[article,incollection,unpublished,phdthesis]{title}%
    {\usebibmacro{string+doiurlisbn}{\mkbibquote{#1}}}
   % {\usebibmacro{string+doiurlisbn}{\mkbibquote{#1.\isdot}}}

\renewcommand*{\newunitpunct}{.\space}


%\bibliography{/Users/smueller/Documents/GitHub/literature/muellerlibrary.bib}
\bibliography{/Users/stefan/GitHub/literature/muellerlibrary.bib}


\usepackage{xcolor}
%\definecolor{JournalBlue}{RGB}{0, 12, 146}

\definecolor{JournalBlue}{RGB}{25, 63, 144}

%https://en.wikibooks.org/wiki/LaTeX/Colors
\usepackage[colorlinks=true, linkcolor=JournalBlue, filecolor=black, urlcolor=JournalBlue, pdfborder={0 0 0},citecolor=JournalBlue]{hyperref}%RoyalBlue
%\usepackage[colorlinks]{hyperref}

\clubpenalty = 10000 
\widowpenalty = 10000 
\displaywidowpenalty = 10000

\setlength\parindent{0pt}


\usepackage{titlesec}
\titleformat{\section}
   {\normalfont\Large\bfseries}{\thesection}{1em}{}

   

\begin{document}
	
\singlespacing

\noindent
\adjustbox{valign=t}{\begin{minipage}{0.1\textwidth}% adapt widths of minipages to your needs
\includegraphics[width=\linewidth]{pictures/ucd_logo}
\end{minipage}}%
\hfill%
\adjustbox{valign=t}{\begin{minipage}{0.9\textwidth}\raggedleft
{%\footnotesize
\textbf{Dr.\ Stefan Müller} \\
Assistant Professor and Ad Astra Fellow \\
School of Politics and International Relations\\
University College Dublin \\
Belfield, Dublin 4, Ireland \\
%\Telefon\ + 353\,89\,975\,25\,79 \\
\Letter\ \href{mailto:stefan.mueller@ucd.ie}{\textsf{stefan.mueller@ucd.ie}}\\
\ComputerMouse\ \url{https://muellerstefan.net} \\
}
\end{minipage}}

\singlespacing
\vspace{1cm}

\begin{center}
{\large Year 3 Module; 
Spring Trimester 2020} \\ 
\bigskip

{\Large \textbf{Parties and Party Competition} (\href{https://sisweb.ucd.ie/usis/!W_HU_MENU.P_PUBLISH?p_tag=MODULE&MODULE=POL30830}{POL30830})} 
\bigskip


{\large  {Last update: \today}}\\
\bigskip

%Latest version: \url{https://muellerstefan.net/teaching/2019-autumn-pceppo.pdf}
\end{center}

\vspace{1.5cm}



\hrule
\medskip
% first column
\begin{minipage}[t]{0.5\textwidth}
Term: Spring Trimester 2020 \\
Time: Mon 13:00--13:50 \& Wed 15:00--15:50 \\
Locations:  Mon: L107 LIB; Wed: NTh 1 ART  \\
ECTS: 5.0 \\
Format: Lecture; group work and class discussion
\end{minipage}
%second column
\begin{minipage}[t]{0.49\textwidth}
\begin{flushright}
Convener: Dr.\ Stefan Müller \\
 \href{mailto:stefan.mueller@ucd.ie}{\textsf{stefan.mueller@ucd.ie}} \\
 \url{https://muellerstefan.net} \\
Office:  TBC \\
Office hours: email for appointment
\end{flushright}
\end{minipage}
\medskip
%\vspace{2.5mm}
\hrule 

\section*{Module Description}

How can we  identify differences between party systems, determine party positions, and measure public opinion?  
Do parties keep their promises or are politicians ``pledge breakers''? Are promises in certain policy areas more likely to be fulfilled? In what policy areas do parties differ in terms of their positions and issue emphasis? And do parties respond to changes in public opinion?  

In this module, we first  discuss the main functions of political parties, outline features of representative democracies, and identify ways of measuring public opinion. Next, we assess whether parties keep their promises, whether the ``mandate model of democracy'' is a desirable and realistic mode of political representation, and how existing studies on election pledge fulfilment can be improved. Afterwards, we investigate parties'  willingness and capacity to respond to changes in public opinion. Fourth, we  discuss different approaches of measuring party positions, political ideology, and the salience of policy areas. Based on these methodological approaches, we identify the circumstances under which parties change their positions and issue emphasis. Finally, we briefly discuss  alternative types of political  participation that go beyond representative government and electoral democracy. 


\section*{Learning Outcomes}

\begin{enumerate}
\item Extensive knowledge of central theories of representation,  the mandate model of democracy, and party competition
\item Detailed insights into past and current approaches to study questions about pledge fulfilment, party positions, responsiveness and issue ownership 
\item Critical reading and discussing  complex academic literature and diverse  quantitative and qualitative methodological approaches
\end{enumerate}

\section*{Indicative Module Content}

The following topics will be covered in this course: parties and party systems; the ``mandate model of democracy''; measuring and aggregating public opinion; economic voting; the cost of governing; responsiveness; party competition; party positions, salience, and issue ownership; deliberative and direct democracy 


\section*{Approaches to Teaching and Learning}

\begin{itemize}
\item Active and task-based learning
\item Group work and discussions
\item In-class debates
\item Problem-based learning
\end{itemize}




%\section*{Module Requirements}


\subsection*{Overview of Assessment}

\begin{itemize}
\item 1,000 word response paper: 30\% 
\item 2,500--3,000 word essay from a choice set of questions: 70\%
\end{itemize}


\subsection*{Expectations and Guidelines}
\begin{itemize}


\item Students are expected to read the papers or chapters assigned under \textbf{mandatory readings}. These readings serve as the basis for in-class discussions about the advantages, disadvantages, and applicability of the various approaches to social science questions. 

\item Students are required to submit one \textbf{response paper} (1,000 words) throughout the course, which counts towards 30\% of the grade. By Week 3, everyone will have been assigned a week where they will prepare a response paper. Response papers must be submitted via \href{https://brightspace.ucd.ie/d2l/home}{Brightspace} no later than \textbf{Monday, 9am of the respective week}, meaning that the assignment has to be submitted \textit{before} the texts are discussed in class. Students are required to choose \textit{one} of the required or suggested readings for that week (readings marked with a star may not be used).  Response papers must contain the following two aspects:
\begin{enumerate}
\item Identify either a limitation of the paper (e.g., how a variable is measured, or an unreasonable/unnecessary assumption) or a possible extension. Either way you should have only one argument in these papers.
\item Suggest a possible solution to that limitation or describe how you would carry out the extension. Note that what you propose should be feasible (ideally by you). If, for example, you find the author's data weak, then you should identify better data, or at least propose a plausible way of collecting these data. If you think the method is wrong, explain why and suggest a better one. If the conclusions do not follow from the premises, discuss what conclusions are actually supported. A specific course of action should be outlined. This process might help you down the line in finding a dissertation topic.
I am not interested in a summary of the paper. The idea is for you to try out ideas for future research projects. These short papers are due by the start of class that week at the latest. 
\end{enumerate}


\item Students submit an \textbf{essay}  which counts towards 70\% of the final grade. The essay must not exceed 2,500--3,000 words  (including bibliography, captions, and footnotes) and  will tackle one of the `discussion questions' which will be published in due course. For this essay, you are required to (i) draw on academic literature (articles and/or books) and (ii) properly cite the academic literature you use to prepare your essay. You should attach an \textit{alphabetised}  bibliography to your essay. Students should read beyond the reading list for this essay. The essay must be submitted via \href{https://brightspace.ucd.ie/d2l/home}{Brightspace} as a \texttt{PDF} document before \textbf{24 April 2020 (8:00pm CET)}. More information on the essay will be provided in the seminar. For information on academic writing, I recommend the following sources:

\begin{itemize}
\item \fullcite{dunleavy14}.
\item \fullcite{heard16}.
\end{itemize}

For the essay, I recommend to pay special attention to the following aspects:

\begin{itemize}
\item \textit{Focus on argumentation, demonstrate critical thinking}: Your essay will be judged primarily on your ability to make nuanced arguments and to demonstrate your understanding of the nuances of the arguments presented by the authors discussed in the course and readings that go beyond the syllabus. While you are expected to engage with the material in the course during your essay, a good essay will do so in a creative way where your own voice comes through clearly. This can be done by critically commenting on the arguments of others; creatively combing arguments from others to make a case; and/or presenting your own original arguments in attempting to improve upon shortcomings in the literature that you have identified.
\item \textit{Read deeply, read widely}: Reading deeply is the most important thing for developing your essay. But you should also read widely, consulting sources both within and beyond the syllabus. It is possible to write a great paper by focusing on just a small number of sources. But this is rare enough. As a rule of thumb, well-researched papers usually average between one and two distinct references per double-spaced page. For a 2,500- 3000-word essay, this will amount to approximately 10--15 distinct references to texts that you have read and analysed closely. 
\item \textit{Presentation}: Be attentive to the presentation of your essay, including consistent referencing-style (with page numbers provided), a bibliography, and  a consistent layout. Learning how to deliver well-presented and polished-looking work is part of your undergraduate training and a highly transferable skill. Take it seriously. Poor presentation will result in lost marks. If you require information on proper citation style, please refer to the guidelines of the American Political Science Association: 
\begin{itemize}
\item \fullcite{apsa18}.
\end{itemize}
\end{itemize}
\end{itemize}



Plagiarism is an issue we take very serious here in UCD. Please familiarize yourself with the definition of plagiarism on UCD's website and make sure not to engage in it.



\begin{comment}
\section*{Feedback Strategy}

\begin{itemize}
\item Feedback individually to students, post-assessment
%\item Feedback individually to students, on an activity or draft prior to summative assessment
%\item Group/class feedback, post-assessment
\end{itemize}
\end{comment}


\begin{table}[h] \centering \onehalfspacing
\caption*{Student effort hours}
\begin{tabular}{ l r} 
\toprule
Student effort type &  Hours \\
\midrule
Seminars & 22 \\
Autonomous Student Learning  & 103 \\
\textbf{Total} & \textbf{125} \\
\bottomrule
\end{tabular}
\end{table}




\section*{Introductory Readings}

The seminar does not build on a single text book, but relies mostly on papers and chapters of books. For  a general overview of the course content, I recommend the following books:

\begin{itemize}
\item \fullcite{powell00}.
\item \fullcite{dalton11}.
\item \fullcite{sartori05}.
%\item \fullcite{gallagher11}.
\item \fullcite{naurin19}.
\item \fullcite{mair13}.
\end{itemize}




\section*{Technical Background and Prerequisites}

The course requires knowledge of general approaches and theories of political science. The following books provide very good introductions to research design and applied quantitative methods.

\subsection*{Research Design and Quantitative Methods}
\begin{itemize}
%\item \fullcite{king94}.
%\item \fullcite{gerring01}.
\item \fullcite{kellstedt19}.
\item \fullcite{imai17}.
%\item \fullcite{wickham17}.
\end{itemize}


\section*{Syllabus Modification Rights}

I reserve the right to reasonably alter the elements of the syllabus at any time by adjusting the reading list to keep pace with the course schedule. Moreover, I may change the content of specific sessions depending on the participants' prior knowledge and research interests.



\newpage

\tableofcontents

\section{Week 1: Conceptualising Representative Democracy (20--24 January 2020)}

\begin{itemize}
\renewcommand\labelitemi{--}
\item Expectations
\item Discussion of syllabus
\item What are the main differences between the majoritarian and proportional visions of democracy?
\end{itemize}


\subsubsection*{Mandatory Readings}
\begin{itemize}
\item \fullcite{mair05}.
\item \fullcite[ch. 1--2]{powell00}.
\end{itemize}


\section{Week 2: Parties and Party Systems (27--31 January 2020)}

\begin{itemize}
\renewcommand\labelitemi{--}
\item  What are political parties?
\item What does Lijphart mean by the Westminster Model of Democracy and the Consensus Model of Democracy?
\item How can we distinguish between different types of democracies?
\end{itemize}

\subsubsection*{Mandatory Readings}
\begin{itemize}
%\item \fullcite{boix07}.
\item \fullcite[ch. 1--3]{lijphart12}.
\item \fullcite{katz95}
%\item \fullcite[Kapitel 1]{dalton11}.
\end{itemize}



\subsubsection*{Optional}
\begin{itemize}
\item \fullcite{katz09}.
\end{itemize}



\section{Week 3: Parties and Electoral Competition (3--7 February 2020)}

\begin{itemize}
\renewcommand\labelitemi{--}
\item  What do we mean by political cleavages?
\item How have political cleavages shaped party competition? 
\item How do political cleavages shape party competition and policy outcomes?
\end{itemize}


\subsubsection*{Mandatory Readings}

\begin{itemize}
\item \fullcite{dalton96}.
\item \fullcite{amorimneto97}.
\end{itemize}



\subsubsection*{Optional}
\begin{itemize}
\item \fullcite{powell06}.
\item \fullcite{golder14}.
\item \fullcite{carey95}.
\item \fullcite{iversen06}.
\end{itemize}



%Making Votes Count: Strategic Coordination in the World's Electoral Systems, Cambridge, Chapters 1, 2, 3
%Iversen, T. and Soskice, D. (2006). Electoral Institutions and the Politics of Coalitions: Why Some Democracies Redistribute More than Others. American Political Science Review, 100(2):165–181




\section{Week 4:  Party Competition (10--14 February 2020)}

\begin{itemize}
\renewcommand\labelitemi{--}
\item What goals do parties and politicians pursue?
\item How do parties compete with each other, and how can we measure party competition?
\end{itemize}

\subsubsection*{Mandatory Readings}
\begin{itemize}
\item \fullcite{strom90}.
\item \fullcite{somertopcu15}.
%\item \fullcite{adams09b}. % BJPS
\end{itemize}

\subsubsection*{Optional}
\begin{itemize}
\item \fullcite{stokes63}.
\item \fullcite{tavits07}.
\item \fullcite{boehmelt16}.
\item \fullcite{mcelroy17}.
%\item \fullcite{wagner14}.
%\item \fullcite{spoon15}.
\end{itemize}



\section{Week 5: Parties in Government and Challenger Parties  (17--21 February 2020)}

\begin{itemize}
\renewcommand\labelitemi{--}
\item How do incumbent parties react to challenger parties or new parties?
\item Why do parties join a coalition? And which types of coalitions are most likely to be formed?
\item What happens when populist parties enter parliament?
\end{itemize}

\subsubsection*{Mandatory Readings}
\begin{itemize}
\item \fullcite{kluever19}.
\item \fullcite{abouchadi20}.
\end{itemize}


\subsubsection*{Optional}
\begin{itemize}
\item \fullcite{warwick06}.e
\item \fullcite{ecker17}.
\item \fullcite{martin01}.
\item \fullcite{bischof20}.
\end{itemize}


%Does winning pay? Electoral success and government formation in 15 West European countries




\section{Week 6:  Mandate Model of Democracy (24--28 February 2020)}


\begin{itemize}
\renewcommand\labelitemi{--}
\item What is the `democratic mandate'? 
\item How we measure campaign promises/pledges?
\item Do parties fulfil their promises?
\end{itemize}

\subsubsection*{Mandatory Readings}
\begin{itemize}
\item \fullcite{mansbridge03}.
\item \fullcite{thomson17}.
\end{itemize}

\subsubsection*{Optional}
\begin{itemize}
\item \fullcite{manin99}.* (not suitable for response paper)
\item \fullcite{thomson16b}.
\item \fullcite{matthiess20}.
\item \fullcite{thomson18}.
\end{itemize}




\section{Week 7:   Measuring Party Positions and Issue Salience (2--6 March 2020)}


\begin{itemize}
\renewcommand\labelitemi{--}
\item What are differences between positions, salience, and issue ownership?
\item How can we measure latent policy positions? 
\item What are methodological difficulties when measuring party positions?
\end{itemize}

\subsubsection*{Mandatory Readings}
\begin{itemize}
\item \fullcite{laver14}.
\item \fullcite{budge15}.
\end{itemize}


\subsubsection*{Optional/Presentations}
\begin{itemize}
\item \fullcite{mikhaylov12}.
\item \fullcite{proksch10}.
\item \fullcite{grimmer13}.*  (not suitable for response paper)
\end{itemize}


\section{Saturday, 9 March--Sunday, 22 March: Reading Weeks}


%\begin{comment}
\section{Week 8:  Responsiveness (23--27 March 2020)}

\begin{itemize}
\renewcommand\labelitemi{--}
\item What is democratic responsiveness?
\item Do parties and parties and politicians react to public opinion?  And from a normative perspective, should political actors change their positions and policies depending on citizens' preferences?
\end{itemize}

\subsubsection*{Mandatory Readings}
\begin{itemize}
\item \fullcite{wlezien95}.
\item \fullcite{kluever16}.
\end{itemize}

\subsubsection*{Optional/Presentations}
\begin{itemize}
\item \fullcite{powell04b}.* (not suitable for response paper)
%\item \fullcite{eulau77}.
\item \fullcite{page83}.
\item \fullcite{stimson95}.
%\item \fullcite{soroka10}.
\end{itemize}
%\end{comment}


\section{Week 9: Economic Voting and the Cost of Governing (30 March--3 April 2020)}


\begin{itemize}
\renewcommand\labelitemi{--}
\item What is democratic accountability?
\item Why do government parties  regularly lose public support at the next election?
\end{itemize}

\subsubsection*{Mandatory Readings}

\begin{itemize}
\item \fullcite{anderson00}.
\item \fullcite{muellerlouwerse}.
\end{itemize}

\subsubsection*{Optional}
\begin{itemize}
\item \fullcite{marsh10}.
\item \fullcite{wlezien17b}.
\item \fullcite{herzog15}.
\end{itemize}

 
\section{Week 10: The (Ir)Rational Voter (6--10 April 2020)}


\begin{itemize}
\renewcommand\labelitemi{--}
\item Are voters rational decision-makers, as assumed in many theories of representation? 
\item Under what circumstances do voters behave (ir)rationally? What are consequences of irrational voting behaviour on political processes and decisions?
\end{itemize}


\subsubsection*{Mandatory Readings}


 \begin{itemize}
\item \fullcite[ch. 5]{achen16}.
\item \fullcite{healy10}.
\item \fullcite{muellerkneafsey}.
\end{itemize}

\subsubsection*{Optional}
\begin{itemize}
\item \fullcite{sances17}.
%\item \fullcite{reif80}.
\item \fullcite{fowler18}.
%\item \fullcite{healy13}.
\end{itemize}



\section{Week 11: Alternative Forms of Participation (13--17 April 2020)}

\begin{itemize}
\renewcommand\labelitemi{--}
\item What are problems associated with representative politics?
\item What other forms of participation exist? What are their strengths and weaknesses?
\item How can we combine representative politics with these alternative forms of participation?
\end{itemize}

\subsubsection*{Mandatory Readings}
\begin{itemize}
\item \fullcite{mair13}.
\item \fullcite{dryzek19}.
\item \fullcite{farrell14}.
\end{itemize}


\subsubsection*{Optional}
\begin{itemize}
\item \fullcite[ch. 1--3]{altman11}.
\item \fullcite{bowler07}.
\item \fullcite{colombo18}.
\item \fullcite{hug09}.* (not suitable for response paper)
\end{itemize}


%\newpage
\sloppy
\renewcommand*{\bibfont}{\small}

\setlength{\bibitemsep}{0.2em} % increase space between references
\printbibliography

\end{document}

\bigskip

%\begin{center}
%Last updated: \today
%\end{center}




