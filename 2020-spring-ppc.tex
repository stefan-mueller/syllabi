\documentclass[abstract=on,parskip=full,headings=standardclasses,fontsize=11pt,paper=a4]{scrartcl}
\usepackage[paper=a4paper,left=21mm,right=21mm,top=25mm,bottom=25mm]{geometry}
\usepackage[utf8]{inputenc}
\usepackage[T1]{fontenc}
\usepackage[english]{babel}

\usepackage{adjustbox}
%\usepackage{amsmath}
\usepackage{graphicx}
%\usepackage{fullpage}
\usepackage{authblk}
\usepackage{setspace}
\usepackage{caption}
\usepackage{booktabs}
\usepackage{url}
\usepackage{comment}
\urlstyle{sf}
\usepackage{lmodern}
\usepackage[parfill]{parskip}
%\usepackage{url}
%\urlstyle{same}
\usepackage[small]{titlesec}
\usepackage{marvosym}

\setcounter{secnumdepth}{0}

\addto\captionsenglish{% Replace "english" with the language you use
  \renewcommand{\contentsname}%
    {Course Structure (Autumn Term 2019)}%
}



%\deffootnote[10pt]{10pt}{10pt}{\makebox[15pt][l]{\thefootnotemark\hspace{10pt}}}

% Use authoryear-comp to create: (Müller 2015, 2016) instead of (Müller 2015; Müller 2016)

% posscite function

\usepackage[style=authoryear-comp,
    maxcitenames=99,
    maxbibnames=99,
    doi=false,
    %sorting=ynt,
    firstinits=false,
    isbn=false,
    date=short,
    dashed=false,
    url=false,
    sortcites=false,
    backend=bibtex]{biblatex}

\makeatletter
\def\blx@maxline{77}
\makeatother


\DeclareNameFormat{labelname:poss}{% Based on labelname from biblatex.def
  \nameparts{#1}% Not needed if using Biblatex 3.4
  \ifcase\value{uniquename}%
    \usebibmacro{name:family}{\namepartfamily}{\namepartgiven}{\namepartprefix}{\namepartsuffix}%
  \or
    \ifuseprefix
      {\usebibmacro{name:first-last}{\namepartfamily}{\namepartgiveni}{\namepartprefix}{\namepartsuffixi}}
      {\usebibmacro{name:first-last}{\namepartfamily}{\namepartgiveni}{\namepartprefixi}{\namepartsuffixi}}%
  \or
    \usebibmacro{name:first-last}{\namepartfamily}{\namepartgiven}{\namepartprefix}{\namepartsuffix}%
  \fi
  \usebibmacro{name:andothers}%
  \ifnumequal{\value{listcount}}{\value{liststop}}{'s}{}}
\DeclareFieldFormat{shorthand:poss}{%
  \ifnameundef{labelname}{#1's}{#1}}
\DeclareFieldFormat{citetitle:poss}{\mkbibemph{#1}'s}
\DeclareFieldFormat{label:poss}{#1's}
\newrobustcmd*{\posscitealias}{%
  \AtNextCite{%
    \DeclareNameAlias{labelname}{labelname:poss}%
    \DeclareFieldAlias{shorthand}{shorthand:poss}%
    \DeclareFieldAlias{citetitle}{citetitle:poss}%
    \DeclareFieldAlias{label}{label:poss}}}
\newrobustcmd*{\posscite}{%
  \posscitealias%
  \textcite}
\newrobustcmd*{\Posscite}{\bibsentence\posscite}
\newrobustcmd*{\posscites}{%
  \posscitealias%
  \textcites}

\renewbibmacro{in:}{} % no "in" before article

\renewcommand*{\bibpagespunct}{\addcomma\space} % ":" instead of pp
\DeclareFieldFormat{pages}{#1}

% Colon after title
\renewcommand{\subtitlepunct}{\addcolon\addspace }

% Colon instead of pp in references
\renewcommand*{\bibpagespunct}{\addcolon\space}
\DeclareFieldFormat{pages}{#1}

% Colon after name in text
\renewcommand*{\postnotedelim}{\addcolon\space}
\DeclareFieldFormat{postnote}{#1}
\DeclareFieldFormat{multipostnote}{#1}

% Remove brackets around year in bibliography
\usepackage{xpatch,filecontents}

\xpatchbibmacro{date+extrayear}{%
  \printtext[parens]%
}{%
  \setunit*{\addperiod\space}%
  \printtext%
}{}{}

% Supresses URL accessed day
\AtEveryBibitem{%
  \ifentrytype{electronic}
    {}
    {\clearfield{urlyear}\clearfield{urlmonth}\clearfield{urlday}}}
%\DefineBibliographyStrings{english}{%
%urlseen = {Accessed},}

\renewbibmacro*{volume+number+eid}{% number of journal in brackets
 \printfield{volume}%
  %\setunit*{\adddot}% DELETED
  \setunit*{\addnbthinspace}% NEW (optional); there's also \addnbthinspace
  \printfield{number}%
  \setunit{\addcomma\space}%
  \printfield{eid}}
\DeclareFieldFormat[article]{number}{\mkbibparens{#1}}



% Change edition field

\DeclareFieldFormat{edition}%
                   {\ifinteger{#1}%
                    {\mkbibordedition{#1}\addthinspace{}edition}%
                    {#1\isdot}}

% New command to show doi, or url or isbn or issn field
% http://tex.stackexchange.com/questions/48400/biblatex-make-title-hyperlink-to-dois-url-or-isbn
\newbibmacro{string+doiurlisbn}[1]{%
  \iffieldundef{doi}{%
    \iffieldundef{url}{%
      \iffieldundef{isbn}{%
        \iffieldundef{issn}{%
          #1%
        }{%
          \href{http://books.google.com/books?vid=ISSN\thefield{issn}}{#1}%
        }%
      }{%
        \href{http://books.google.com/books?vid=ISBN\thefield{isbn}}{#1}%
      }%
    }{%
      \href{\thefield{url}}{#1}%
    }%
  }{%
    \href{http://dx.doi.org/\thefield{doi}}{#1}%
  }%
}

% Necessary to remove dot after question mark in title
%\newcommand{\killpunct}[1]{}    

% Make full stop after title and before quotation marks in title field
\DeclareFieldFormat{title}{\usebibmacro{string+doiurlisbn}{\mkbibemph{#1}}}
\DeclareFieldFormat[article,incollection,unpublished,phdthesis]{title}%
    {\usebibmacro{string+doiurlisbn}{\mkbibquote{#1}}}
   % {\usebibmacro{string+doiurlisbn}{\mkbibquote{#1.\isdot}}}

\renewcommand*{\newunitpunct}{.\space}


\bibliography{/Users/smueller/Documents/GitHub/literature/muellerlibrary.bib}
%\bibliography{/Users/stefan/GitHub/literature/muellerlibrary.bib}


\usepackage{xcolor}
\definecolor{JournalBlue}{RGB}{0, 12, 146}
%https://en.wikibooks.org/wiki/LaTeX/Colors
\usepackage[colorlinks=true, linkcolor=JournalBlue, filecolor=black, urlcolor=JournalBlue, pdfborder={0 0 0},citecolor=JournalBlue]{hyperref}%RoyalBlue
%\usepackage[colorlinks]{hyperref}

\clubpenalty = 10000 
\widowpenalty = 10000 
\displaywidowpenalty = 10000

\setlength\parindent{0pt}


\usepackage{titlesec}
\titleformat{\section}
   {\normalfont\large\bfseries}{\thesection}{1em}{}

   

\begin{document}
	
\singlespacing

\noindent
\adjustbox{valign=t}{\begin{minipage}{0.1\textwidth}% adapt widths of minipages to your needs
\includegraphics[width=\linewidth]{pictures/ucd_logo}
\end{minipage}}%
\hfill%
\adjustbox{valign=t}{\begin{minipage}{0.9\textwidth}\raggedleft
{%\footnotesize
\textbf{Dr. Stefan Müller} \\
Assistant Professor and Ad Astra Fellow \\
School of Politics and International Relations\\
University College Dublin \\
Belfield, Dublin 4, Ireland \\
%\Telefon\ + 353\,89\,975\,25\,79 \\
\Letter\ \href{mailto:stefan.mueller@ucd.ie}{\textsf{stefan.mueller@ucd.ie}}\\
\ComputerMouse\ \url{https://muellerstefan.net} \\
}
\end{minipage}}

\singlespacing
\vspace{1cm}

\begin{center}
{\large Year 3 Module; 
Spring Trimester 2020} \\ 
\bigskip

{\Large \textbf{Parties and Party Competition} (\href{https://sisweb.ucd.ie/usis/!W_HU_MENU.P_PUBLISH?p_tag=MODULE&MODULE=POL30830}{POL30830})} 
\bigskip


{\large  {Last update: \today}}\\
\bigskip

%Latest version: \url{https://muellerstefan.net/teaching/2019-autumn-pceppo.pdf}
\end{center}

\vspace{1.5cm}



\hrule
\medskip
% first column
\begin{minipage}[t]{0.5\textwidth}
Term: Spring Trimester 2020 \\
Time: Mon 13:00--13:50 \& Wed 15:00--15:50 \\
Location:  TBC \\
ECTS: 5.0 \\
Format: Lecture; group work and class discussion
\end{minipage}
%second column
\begin{minipage}[t]{0.49\textwidth}
\begin{flushright}
Convener: Stefan Müller \\
 \href{mailto:stefan.mueller@ucd.ie}{\textsf{stefan.mueller@ucd.ie}} \\
 \url{https://muellerstefan.net} \\
Office:  TBC \\
Office hours: email for appointment
\end{flushright}
\end{minipage}
\medskip
%\vspace{2.5mm}
\hrule 

\section*{Module Description}

How can we  identify differences between party systems, determine party positions, and measure public opinion?  
Do parties keep their promises or are politicians ``pledge breakers''? Are promises in certain policy areas more likely to be fulfilled? In what policy areas do parties differ in terms of their positions and issue emphasis? And do parties respond to changes in public opinion?  

In this module, we first  discuss the main functions of political parties, outline features of representative democracies, and identify ways of measuring public opinion. Next, we assess whether parties keep their promises, whether the ``mandate model of democracy'' is a desirable and realistic mode of political representation, and how existing studies on election pledge fulfilment can be improved. Afterwards, we investigate parties'  willingness and capacity to respond to changes in public opinion. Fourth, we  discuss different approaches of measuring party positions, political ideology, and the salience of policy areas. Based on these methodological approaches, we identify the circumstances under which parties change their positions and issue emphasis. Finally, we briefly discuss  alternative types of political  participation that go beyond representative government and electoral democracy. 


\section*{Learning Outcomes}

\begin{enumerate}
\item Extensive knowledge of central theories of representation,  the mandate model of democracy, and party competition
\item Detailed insights into past and current approaches to study questions about pledge fulfilment, party positions, responsiveness and issue ownership 
\item Critical reading and discussing  complex academic literature and diverse  quantitative and qualitative methodological approaches
\end{enumerate}

\section*{Indicative Module Content}

The following topics will be covered in this course: parties and party systems; the ``mandate model of democracy''; measuring and aggregating public opinion; economic voting; the cost of governing; responsiveness; party competition; party positions, salience, and issue ownership; deliberative and direct democracy 


\section*{Approaches to Teaching and Learning}

\begin{itemize}
\item Active and task-based learning
\item Group work and discussions
\item In-class debates
\item Problem-based learning
\end{itemize}

\section*{Assessment}

\begin{itemize}
\item 1,000 word response paper: 30\% 
\item 2,500--3,000 word essay from a choice set of questions: 70\%
\end{itemize}

\begin{comment}
\section*{Feedback Strategy}

\begin{itemize}
\item Feedback individually to students, post-assessment
%\item Feedback individually to students, on an activity or draft prior to summative assessment
%\item Group/class feedback, post-assessment
\end{itemize}
\end{comment}


\begin{table}[h] \centering \onehalfspacing
\caption*{Overview of deadlines}
\begin{tabular}{ l l l} 
\toprule
Date &  Time & Assignment \\
\midrule
Friday, 20 March 2020 & 8:00pm CET &  Response Paper (30\%)  \\
TBC  & 8:00pm CET  & Essay (70\%) \\
\bottomrule
\end{tabular}
\end{table}



\section*{Introductory Readings}

The seminar does not build on a single text book, but relies mostly on papers and chapters of books. For  a general overview of the course content, I recommend the following books:

\begin{itemize}
\item \fullcite{powell00}.
\item \fullcite{dalton11}.
%\item \fullcite{gallagher11}.
\item \fullcite{naurin19}.
\item \fullcite{volkens13}.
\end{itemize}




\section*{Technical Background and Prerequisites}

The course requires knowledge of general approaches and theories of political science. The following books provide very good introductions to research design and applied quantitative methods.

\subsection*{Research Design and Quantitative Methods}
\begin{itemize}
%\item \fullcite{king94}.
%\item \fullcite{gerring01}.
\item \fullcite{kellstedt19}.
\item \fullcite{imai17}.
%\item \fullcite{wickham17}.
\end{itemize}

\subsection*{Academic Writing}
\begin{itemize}
\item \fullcite{heard16}.
\end{itemize}


\section*{Syllabus Modification Rights}

I reserve the right to reasonably alter the elements of the syllabus at any time by adjusting the reading list to keep pace with the course schedule. Moreover, I may change the content of specific sessions depending on the participants' prior knowledge and research interests.


\section*{Expectations and Grading}


\begin{itemize}
\item Students are expected to read the papers or chapters assigned under \textbf{Mandatory Readings}. These readings serve as the basis for in-class discussions about the advantages, disadvantages, and applicability of the various approaches to social science questions. I also add optional readings which will be presented by students during their in-class presentation (see details below). 

\item Students will prepare a  \textbf{Presentation} of one of the optional readings. This presentation counts 40\% towards the grade for this term. Dates and texts for presentations will be assigned in the third week of the seminar. The presentation includes a brief and concise discussion of the paper or book, with particular reference to the puzzle, research question, hypotheses, and results. The main part of the presentation should be devoted to a critical assessment of the paper. What open questions remain and how has subsequent research addressed these questions? What are weaknesses of the methods or case selection strategy? Are results internally and externally valid and generalisable? And how would you improve or extend the study? The presentations will take in weeks 6--10.

\item Students also submit a \textbf{Research Proposal} which counts towards 60\% of the final grade. The research proposal must not exceed 4,000 words  (including bibliography, captions, and footnotes).  The proposal  should identify a  research question, a discussion of the variation to be explained, and the importance of the research question. Moreover, the students should specify observable implications, the measurement and conceptualisation of the dependent and main independent variable, and propose a methodological approach to analyse this question. More details on these aspects and the research design will be provided throughout the seminar. The research design must be submitted via \href{https://lms.uzh.ch}{OLAT} as a \texttt{PDF} document before \textbf{December 6, 2019 (8:00pm CET)}. 
\end{itemize}



\newpage

\tableofcontents

\section{Week 1: Conceptualising Democracy (20--24 January 2020)}

\begin{itemize}
\renewcommand\labelitemi{--}
\item Expectations
\item Discussion of syllabus
\end{itemize}


\subsubsection*{Mandatory Readings}
\begin{itemize}
\item \fullcite[ch. 1--2]{powell00}.
\end{itemize}


\section{Week 2: Parties and Party Systems (27--31 January 2020)}

\begin{itemize}
\renewcommand\labelitemi{--}
\item  What are political parties?
\item What does Lijphart mean by the Westminster Model of Democracy and the Consensus Model of Democracy?
\item How can we distinguish between different types of democracies?
\end{itemize}

\subsubsection*{Mandatory Readings}
\begin{itemize}
%\item \fullcite{boix07}.
\item \fullcite[ch. 1--3]{lijphart12}.
\item \fullcite{katz95}
%\item \fullcite[Kapitel 1]{dalton11}.
\end{itemize}



\subsubsection*{Optional}
\begin{itemize}
\item \fullcite{katz09}.
\item \fullcite{vanbiezen12}.
\end{itemize}



\section{Week 3: Parties and Electoral Competition (3--7 February 2020)}

\subsubsection*{Mandatory Readings}

Making Votes Count: Strategic Coordination in the World's Electoral Systems, Cambridge, Chapters 1, 2, 3


Electoral Institutions, Cleavage Structures, and the Number of Parties


Iversen, T. and Soskice, D. (2006). Electoral Institutions and the Politics of Coalitions: Why Some Democracies Redistribute More than Others. American Political Science Review, 100(2):165–181

something on ireland (?)


\section{Week 4: Parties in Government and Challenger Parties (10--14 February 2020)}


\subsubsection*{Mandatory Readings}
\begin{itemize}
\item \fullcite{kluever19}.
\end{itemize}



Does winning pay? Electoral success and government formation in 15 West European countries

bischof party politics


\section{Week 5:  Party Competition (17--21 February 2020)}

\begin{itemize}
\renewcommand\labelitemi{--}
\item What goals do parties and politicians pursue?
\item How do parties compete with each other, and how can we measure party competition?
\end{itemize}

\subsubsection*{Mandatory Readings}
\begin{itemize}
\item \fullcite{strom90}.
\item \fullcite{somertopcu15}.
%\item \fullcite{adams09b}. % BJPS
\end{itemize}

\subsubsection*{Optional/Presentations}
\begin{itemize}
\item \fullcite{stokes63}.
\item \fullcite{greenpedersen07}.
\item \fullcite{tavits07}.
\item \fullcite{boehmelt16}.
%\item \fullcite{wagner14}.
%\item \fullcite{spoon15}.
\end{itemize}


\section{Week 6:  Mandate Model of Democracy (24--28 February 2020)}


\begin{itemize}
\renewcommand\labelitemi{--}
\item What is the `democratic mandate'? 
\item How we measure campaign promises/pledges?
\item Do parties fulfil their promises?
\end{itemize}

\subsubsection*{Mandatory Readings}
\begin{itemize}
\item \fullcite[29--40]{manin99}.
\item \fullcite{thomson17}.
\end{itemize}

\subsubsection*{Optional/Presentations}
\begin{itemize}
\item \fullcite{thomson16b}.
\end{itemize}




\section{Week 7:   Measuring Party Positions and Issue Salience (2--6 March 2020)}


\begin{itemize}
\renewcommand\labelitemi{--}
\item What are differences between positions, salience, and issue ownership?
\item How can we measure latent policy positions? 
\item What are methodological difficulties when measuring party positions?
\end{itemize}

\subsubsection*{Mandatory Readings}
\begin{itemize}
\item \fullcite{laver14}.
\item \fullcite{budge15}.
\end{itemize}


\subsubsection*{Optional/Presentations}
\begin{itemize}
\item \fullcite{leinaweaver16}.
\item \fullcite{mikhaylov12}.
\item \fullcite{somertopcu15}.
\item \fullcite{bischof19}.
\end{itemize}


\begin{comment}
\section{Week 8:  Responsiveness (23--27 March 2020)}

\begin{itemize}
\renewcommand\labelitemi{--}
\item What are the differences between accountability and responsiveness?
\item Do parties and politicians react to public opinion? 
\end{itemize}

\subsubsection*{Mandatory Readings}
\begin{itemize}
\item \fullcite{wlezien95}.
\item \fullcite{kluever16}.
\end{itemize}

\subsubsection*{Optional/Presentations}
\begin{itemize}
\item \fullcite{powell04b}.
%\item \fullcite{eulau77}.
\item \fullcite{page83}.
\item \fullcite{stimson95}.
%\item \fullcite{soroka10}.
\end{itemize}
\end{comment}


\section{Week 9: Economic Voting and the Cost of Governing (30 March--3 April 2020)}


\begin{itemize}
\renewcommand\labelitemi{--}
\item What is democratic accountability?
\item Why do government parties  regularly lose public support at the next election?
\end{itemize}

\subsubsection*{Mandatory Readings}

\begin{itemize}
\item \fullcite{anderson00}.
\item \fullcite{muellerlouwerse}.
\end{itemize}



 
\section{Week 10: The (Ir)Rational Voter (6--10 April 2020)}


\subsubsection*{Mandatory Readings}


 \begin{itemize}
\item \fullcite[ch. 5]{achen16}.
\item \fullcite{healy10}.
\item \fullcite{muellerkneafsey}.
\end{itemize}

\subsubsection*{Optional/Presentations}
\begin{itemize}
\item \fullcite{sances17}.
%\item \fullcite{reif80}.
\item \fullcite{fowler18}.
%\item \fullcite{healy13}.
\end{itemize}



\section{Week 11: Alternative Forms of Participation (13--17 April 2020)}

\begin{itemize}
\renewcommand\labelitemi{--}
\item What is quantitative text analysis?
\item What is a text corpus, tokenisation, and a document-feature matrix?
\end{itemize}

\subsubsection*{Mandatory Readings}
\begin{itemize}
\item \fullcite{mair13}.
\item \fullcite{colombo18}.
\item nature paper
\end{itemize}


\end{document}

%\newpage
\sloppy
\renewcommand*{\bibfont}{\small}

\setlength{\bibitemsep}{0.2em} % increase space between references
%\printbibliography

\bigskip

%\begin{center}
%Last updated: \today
%\end{center}




