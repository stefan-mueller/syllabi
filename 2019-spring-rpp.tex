\documentclass[abstract=on,parskip=full,headings=standardclasses,fontsize=11pt,paper=a4]{scrartcl}
\usepackage[paper=a4paper,left=25mm,right=25mm,top=20mm,bottom=25mm]{geometry}
\usepackage[utf8]{inputenc}
\usepackage[T1]{fontenc}
\usepackage[ngerman]{babel}

\usepackage{adjustbox}
%\usepackage{amsmath}
\usepackage{graphicx}
%\usepackage{fullpage}
\usepackage{authblk}
\usepackage{caption}
\usepackage{setspace}
\usepackage{lmodern}
\usepackage{url}
\urlstyle{sf}
%usepackage{titlesec}
\usepackage{marvosym}
\usepackage{booktabs}

\input{mueller_bib_custom_ger.tex}

\bibliography{/Users/smueller/Documents/GitHub/literature/muellerlibrary.bib}

\setcounter{secnumdepth}{0}


\usepackage{titlesec}
\titleformat{\subsubsection}
   {\normalfont\normalsize\itshape}{\thesubsubsection}{1em}{}
   
   
\usepackage{xcolor}
\definecolor{JournalBlue}{RGB}{0, 12, 146}
%https://en.wikibooks.org/wiki/LaTeX/Colors
\usepackage[colorlinks=true, linkcolor=JournalBlue, filecolor=black, urlcolor=JournalBlue, pdfborder={0 0 0},citecolor=JournalBlue]{hyperref}%RoyalBlue
%\usepackage[colorlinks]{hyperref}

\makeatletter
\def\blx@maxline{77}
\makeatother

\usepackage[parfill]{parskip}
\setlength\parindent{0pt}


\begin{document}
	
\singlespacing

\noindent
\adjustbox{valign=t}{\begin{minipage}{0.38\textwidth}% adapt widths of minipages to your needs
\includegraphics[width=\linewidth]{pictures/uzh_logo_de}
\end{minipage}}%
\hfill%
\adjustbox{valign=t}{\begin{minipage}{0.62\textwidth}\raggedleft
{%\footnotesize
\textbf{Stefan Müller} \\
Assistent \\
Lehrstuhl für Policy-Analyse \\
Universität Zürich \\
%\Letter\ \href{mailto:mullers@tcd.ie}{\textsf{mullers@tcd.ie}} \\
\url{https://muellerstefan.net} \\
}
\end{minipage}}

\singlespacing
\vspace{1cm}

\begin{center}
{\large Wahlmodul \href{https://studentservices.uzh.ch/uzh/anonym/vvz/index.html#/details/2018/004/E/50926420}{615251}} \\ 
\medskip
{\Large \textbf{Repräsentation, Parteienwettbewerb und Policy-Analyse}} \\
\bigskip
{\large  Entwurf (letzte Aktualisierung: \today)}

Neueste Version: \url{https://muellerstefan.net/teaching/2019-spring-rpp.pdf}
\end{center}

\vspace{1.5cm}

\hrule
\medskip
% first column
\begin{minipage}[t]{0.5\textwidth}
Semester: Frühjahrssemester 2019 \\
Zeit: Mittwoch, 10:15--12:00 \\
Veranstaltungsraum: \\
ECTS: 6.0
\end{minipage}
%second column
\begin{minipage}[t]{0.5\textwidth}
\begin{flushright}
Dozent: Stefan Müller \\
Raum: \\
Sprechstunde: \\
E-Mail: \textsf{\href{mailto:mueller@ipz.uzh.ch}{mueller@ipz.uzh.ch}}
\end{flushright}
\end{minipage}
\medskip
\vspace{2.5mm}
\hrule 

\section*{Kursbeschreibung}

Inwiefern unterscheiden sich Parteien inhaltlich? Erfüllen Parteien ihre Wahlversprechen? Unter welchen Umständen reagieren PolitikerInnen auf Änderungen in der öffentlichen Meinung? Und lernen Parteien voneinander? Antworten auf diese Fragen sind unverzichtbar, um gegenwärtige politische Debatten verstehen und einordnen zu können. Dieser Kurs ermöglicht einen systematischen Einblick in die wissenschaftliche Literatur über den Parteienwettbewerb, Repräsentationsfunktionen und öffentliche Politiken. Zu Beginn des Seminars werden zentrale Begriffe wie Repräsentation, Responsivität, Parteienwettbewerb, Wahlversprechen und Diffusion definiert und verknüpft. Daraufhin werden die Beziehungen zwischen Parteien und WählerInnen in den Blick genommen, ehe wir uns der Frage widmen, inwieweit Parteipositionen und konkrete Politiken von Diffusionsprozessen und Umfrageergebnissen beeinflusst werden. Ausserdem werden wir untersuchen, wie bestehende Studien diese Begriffe konzeptualisieren und welche Schwierigkeiten sich bei der Messung derart vielschichtiger Konzepte ergeben können.

\subsection*{Details}

\begin{itemize}
\item BA-Kurs
\item  Kurssprache: Deutsch
\item Benotung: Impulsreferat (10\%); kritische Stellungnahme (20\%); Literaturbericht (70\%)
\end{itemize}



\section*{Einführende Literatur}

\begin{itemize}
\item \fullcite{powell00}.
\item \fullcite{knill15}.
\item \fullcite{gallagher11}.
\item \fullcite{soroka10}.
\item \fullcite{dalton11}.
\item \fullcite{volkens13}.
\end{itemize}

\section*{Lernziele}

\begin{enumerate}
\item Vertiefung der Kenntnisse von zentralen theoretischen Aspekten der Parteien- und Policy-Forschung
\item Detaillierte Kenntnisse von aktuellen wissenschaftlichen Debatten über Repräsentation, Parteienwettbewerb, Responsivität und öffentliche Politiken
\item  Kritisches Lesen, Diskutieren und Aneignen der Inhalte komplexer Fachliteratur und diverser methodischer Vorgehensweisen
\item Konzeption und Durchführung einer kritischen Stellungnahme und eines ausführlichen Literaturberichts
\end{enumerate}

\section*{Erwartungen}

\begin{itemize}
\item Die Studierenden halten ein kurzes \textbf{Impulsreferat} (10\%)  über einen der optionalen Texte, die auf dem Syllabus angegeben sind. Die Verteilung der Referatstexte erfolgt in der zweiten Sitzung. Das Referat soll den Aufsatz oder das Buchkapitel \textit{prägnant und kritisch} (!) bewerten. Meist bauen die Text auf der Pflichtlektüre auf, deren Inhalte \textit{nicht} im Referat wiedergegeben werden sollen. Der Inhalt des gelesen Texts soll kurz wiedergegeben werden. Der Schwepunkt jedoch soll auf einer \textit{kritischen} Einordnung liegen.  Studierende können bis zu vier PowerPoint- oder LaTeX-Slides nutzen, die zur visuellen Unterstützung dienen können. Fals Slides genutzt werden, müssen diese bis 24 Stunden vor Seminarbeginn an \href{mailto:mueller@ipz.uzh.ch}{mueller@ipz.uzh.ch} geschickt werden. 

\item In der \textbf{kritische Stellungnahme} (20\%) diskutieren Studierende ein Papier aus einem englischsprachigen Fachjournal. Der Abgabetermin ist Mittwoch, der \textbf{17. April 2019 um 20:00 Uhr}.   Der Aufbau soll sich hierbei an dem Peer-Review-Prozess orientieren, den wissenschaftliche Papiere vor der Veröffentlichung durchlaufen. Praktische Beispiele werden in den ersten drei Sitzungen präsentiert. Die Stellungnahme umfasst 1000--1200 Wörter und  muss die folgenden Punkte beinhalten: Stärken, Logik, Argumentationsschwächen, Methoden. Studierende sollen selbstständig nach einem passenden Papier suchen, das nicht Teil des Syllabus ist, sich jedoch  mit den Kursinhalten deckt. Der Vorschlag muss mir bis spätestens zum \textbf{29. März 2019}  per E-Mail geschickt werden (\href{mailto:mueller@ipz.uzh.ch}{mueller@ipz.uzh.ch}). Daraufhin werde ich entscheiden, ob das entsprechende Papier besprochen werden kann. Stellungnahmen, die ohne meine vorherige Einwilligung verfasst wurden, werden nicht akzeptiert. 


\item Der \textbf{ausführliche Literaturbericht} (70\%) diskutiert die Literatur über ein Unterthema des Kurses. Der Abgabetermin ist Freitag, der \textbf{7. Juni 2019 um 20:00 Uhr}.  Das Ziel des Literaturberichts ist es \textit{nicht}, die vorhandenen Forschungsergebnisse separat  zusammenzufassen. Stattdessen soll die bisherige Evidenz verglichen,  Querverbindungen zwischen Texten hergestellt und Forschungslücken identifiziert werden. Der Literaturbericht umfasst 3,500--3,800 Wörter (das Literaturverzeichnis ist  nicht Teil dieser Wortanzahl). Bis zum \textbf{10. Mai 2019} müssen die Studierenden mir einen Vorschlag per E-Mail schicken (\href{mailto:mueller@ipz.uzh.ch}{mueller@ipz.uzh.ch}), in dem das Thema des Literaturberichts in einem Satz zusammengefasst wird. Es wird erwartet, dass Studierende in ihrer Recherche weit über die Literatur des Kurses herausgehen. Exzellente Beispiele für Literaturberichte finden sich in den \textit{Annual Reviews of Political Science}: \url{http://www.annualreviews.org/journal/polisci}.


\end{itemize}


\begin{table}[h] \centering \onehalfspacing \small
\caption*{Abgabetermine der schriftlichen Prüfungsleistungen}
\begin{tabular}{ l l l} 
\toprule
Datum &  Zeit & Prüfungsleistung \\
\midrule
Freitag, 29. März  & 20:00 Uhr &  E-Mail mit Vorschlag eines Artikels für die Stellungnahme  \\
Mittwoch, 17. April 2019 & 20:00 Uhr & Einreichung der kritischen Stellungnahme  \\
Freitag, 10. Mai  & 20:00 Uhr & E-Mail mit Vorschlag eines Themas für den Literaturbericht \\
Freitag, 7. Juni 2019 & 20:00 Uhr & Einreichung des ausführlichen Literaturberichts  \\
\bottomrule
\end{tabular}
\end{table}



\tableofcontents

\section{I. Einführung und Überblick}


\subsection{1. Woche: Organisatorisches und Aufbau des Moduls (20.02.)}

%\begin{itemize}
%\item Erwartungen
%\item Besprechung des Syllabus
%\item Hinweise zu Referaten und der Anfertigung von schriftlichen Arbeiten
%\end{itemize}


\subsubsection*{Pflichtlektüre}
\begin{itemize}
\item \fullcite[Kapitel 1]{clarke18}.
%\item \fullcite{best11}.
\item \fullcite{manin99}.
\end{itemize}



\subsection{2. Woche: Überblick  und Definitionen (27.02.)}

%\begin{itemize}
%\item Wie hängen Repräsentation, Parteienwettbewerb und Policy-Analyse zusammen?
%\item Was sind die wichtigsten Theorien?
%\end{itemize}


\subsubsection*{Pflichtlektüre}
\begin{itemize}
\item \fullcite[Kapitel 1]{dalton11}.
\item \fullcite[Kapitel 1--2]{powell00}.
\end{itemize}


\section{II. Repräsentation}

\subsection{3. Woche: Die Mandatstheorie und Wahlversprechen (06.03.)}

\subsubsection*{Pflichtlektüre}
\begin{itemize}
\item \fullcite{mansbridge03}.
\item \fullcite{thomson17}.
\end{itemize}

\subsubsection*{Optional/Referate}
\begin{itemize}
\item \fullcite{royed96}.
\item \fullcite{brouard18}.
\end{itemize}


\subsection{4. Die Messung der Öffentlichen Meinung (13.03.)}

\subsubsection*{Pflichtlektüre}
\begin{itemize}
\item \fullcite{berinsky17}.
\item \fullcite{bowler17}.
\end{itemize}


\subsubsection*{Optional/Referate}
\begin{itemize}
\item \fullcite{chong07b}.
\end{itemize}


%\subsection{4. Woche: Conditional Representation}
%
%canes-wrone 15 table 1 (electoral cycle, salience)

%\subsubsection*{Pflichtlektüre}
%\begin{itemize}
%\item \fullcite{budge15}.
%\item \fullcite{horn17}.
%\end{itemize}



\subsection{5. Verantwortbarkeit und die Kosten des Regierens (20.03.)}

\subsubsection*{Pflichtlektüre}
\begin{itemize}
\item \fullcite[Kapitel 5]{achen16}.
\item \fullcite{fowler18}.
\item \fullcite{healy13}.
\end{itemize}


\subsubsection*{Optional/Referate}
\begin{itemize}
\item \fullcite{sances17}.
\item \fullcite{reif80}.
\item \fullcite{muellerlouwerse}.
\end{itemize}


\subsection{6. Woche: Responsivität (27.03.)}

\subsubsection*{Pflichtlektüre}
\begin{itemize}
\item \fullcite{wlezien95}.
\item \fullcite{powell04b}.
\end{itemize}

\subsubsection*{Optional/Referate}
\begin{itemize}
\item \fullcite{eulau77}.
\item \fullcite{page83}.
\item \fullcite{stimson95}.
\item \fullcite{kluever16}.
%\item \fullcite{soroka10}.
\end{itemize}



\section{III. Parteienwettbeweb}



\subsection{7. Woche: Die Messung von Parteipositionen (03.04.)}

\subsubsection*{Pflichtlektüre}
\begin{itemize}
\item \fullcite{laver14}.
\item \fullcite{hjorth15}.
\item \fullcite{mikhaylov12}.
\end{itemize}


\subsubsection*{Optional/Referate}
\begin{itemize}
\item \fullcite{laver00}.
\item \fullcite{budge13}.
\item \fullcite{benoit16}.
\item \fullcite{lowe11}.
\end{itemize}

\subsection{8. Woche: Sitzung entfällt wegen Konferenz!}


\subsection{9. Woche: Parteienwettbewerb (17.04.)}

\subsubsection*{Pflichtlektüre}
\begin{itemize}
\item \fullcite{stokes63}.
\item \fullcite{strom90}.
\end{itemize}

\subsubsection*{Optional/Referate}
\begin{itemize}
\item \fullcite{greenpedersen07}.
\item \fullcite{tavits07}.
%\item \fullcite{wagner14}.
%\item \fullcite{spoon15}.
\end{itemize}


\subsection{10. Woche: Salienz und Kongruenz (08.05.)}

\subsubsection*{Pflichtlektüre}
\begin{itemize}
\item \fullcite{budge15}.
\item \fullcite{caneswrone15}.
%\item \fullcite{spoon14}.
%\item \fullcite[Kapitel ?]{powell00}.
\end{itemize}

\subsubsection*{Optional/Referate}
\begin{itemize}
\item \fullcite{powell09}.
\item \fullcite{kluever16}.
\item \fullcite{horn17}.
\item \fullcite{budge90}.
\item \fullcite{king93}.
\end{itemize}


\subsection{11. Änderungen von Parteipositionen (15.05.)}

\subsubsection*{Pflichtlektüre}
\begin{itemize}
%\item \fullcite[Kapitel 1--2]{soroka10}.
\item \fullcite{boehmelt16}.
\item \fullcite{bischof19}.
\end{itemize}



\subsubsection*{Optional/Referate}
\begin{itemize}
\item \fullcite{wolkenstein}.
\item \fullcite{adams09}.
\item \fullcite{adams11}.
\item \fullcite{schumacher15}.
\item \fullcite{abouchadi20}.
\end{itemize}



\section{IV. Policy-Analyse}

\subsection{12. Der Policy-Prozess (22.05.)}

\subsubsection*{Pflichtlektüre}
\begin{itemize}
\item \fullcite[Kapitel 1--2]{knill15}.
\end{itemize}
 


\subsection{13. Policy-Diffusion (29.05.)}

\subsubsection*{Pflichtlektüre}
\begin{itemize}
\item \fullcite{graham13}.
\item \fullcite{gilardi16}.
\item \fullcite{elkins05}.
\end{itemize}
 
 
\subsubsection*{Optional/Referate}
\begin{itemize}
\item \fullcite{brooks07b}.
 \item \fullcite{gilardi10}.
\end{itemize}

 


\sloppy
\renewcommand*{\bibfont}{\small}

\setlength{\bibitemsep}{0.2em} % increase space between references
\printbibliography

\bigskip

\begin{center}
Letzte Aktualisierung: \today
\end{center}


\end{document}


