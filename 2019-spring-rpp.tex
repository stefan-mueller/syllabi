\documentclass[abstract=on,parskip=full,headings=standardclasses,fontsize=11pt,paper=a4]{scrartcl}
\usepackage[paper=a4paper,left=25mm,right=25mm,top=20mm,bottom=25mm]{geometry}
\usepackage[utf8]{inputenc}
\usepackage[T1]{fontenc}
\usepackage[ngerman]{babel}

\usepackage{adjustbox}
%\usepackage{amsmath}
\usepackage{graphicx}
%\usepackage{fullpage}
\usepackage{authblk}
\usepackage{setspace}
\usepackage{url}
\usepackage{lmodern}
\usepackage[parfill]{parskip}
\usepackage{url}
\urlstyle{sf}
%usepackage{titlesec}
\usepackage{marvosym}

%\deffootnote[10pt]{10pt}{10pt}{\makebox[15pt][l]{\thefootnotemark\hspace{10pt}}}

% Use authoryear-comp to create: (Müller 2015, 2016) instead of (Müller 2015; Müller 2016)

% posscite function

\usepackage[style=authoryear-comp,
    maxcitenames=99,
    maxbibnames=99,
    doi=false,
    %sorting=ynt,
    firstinits=false,
    isbn=false,
    date=short,
    dashed=false,
    url=false,
    sortcites=false,
    backend=bibtex]{biblatex}

\makeatletter
\def\blx@maxline{77}
\makeatother


\DeclareNameFormat{labelname:poss}{% Based on labelname from biblatex.def
  \nameparts{#1}% Not needed if using Biblatex 3.4
  \ifcase\value{uniquename}%
    \usebibmacro{name:family}{\namepartfamily}{\namepartgiven}{\namepartprefix}{\namepartsuffix}%
  \or
    \ifuseprefix
      {\usebibmacro{name:first-last}{\namepartfamily}{\namepartgiveni}{\namepartprefix}{\namepartsuffixi}}
      {\usebibmacro{name:first-last}{\namepartfamily}{\namepartgiveni}{\namepartprefixi}{\namepartsuffixi}}%
  \or
    \usebibmacro{name:first-last}{\namepartfamily}{\namepartgiven}{\namepartprefix}{\namepartsuffix}%
  \fi
  \usebibmacro{name:andothers}%
  \ifnumequal{\value{listcount}}{\value{liststop}}{'s}{}}
\DeclareFieldFormat{shorthand:poss}{%
  \ifnameundef{labelname}{#1's}{#1}}
\DeclareFieldFormat{citetitle:poss}{\mkbibemph{#1}'s}
\DeclareFieldFormat{label:poss}{#1's}
\newrobustcmd*{\posscitealias}{%
  \AtNextCite{%
    \DeclareNameAlias{labelname}{labelname:poss}%
    \DeclareFieldAlias{shorthand}{shorthand:poss}%
    \DeclareFieldAlias{citetitle}{citetitle:poss}%
    \DeclareFieldAlias{label}{label:poss}}}
\newrobustcmd*{\posscite}{%
  \posscitealias%
  \textcite}
\newrobustcmd*{\Posscite}{\bibsentence\posscite}
\newrobustcmd*{\posscites}{%
  \posscitealias%
  \textcites}

\renewbibmacro{in:}{} % no "in" before article

\renewcommand*{\bibpagespunct}{\addcomma\space} % ":" instead of pp
\DeclareFieldFormat{pages}{#1}

% Colon after title
\renewcommand{\subtitlepunct}{\addcolon\addspace }

% Colon instead of pp in references
\renewcommand*{\bibpagespunct}{\addcolon\space}
\DeclareFieldFormat{pages}{#1}

% Colon after name in text
\renewcommand*{\postnotedelim}{\addcolon\space}
\DeclareFieldFormat{postnote}{#1}
\DeclareFieldFormat{multipostnote}{#1}

% Remove brackets around year in bibliography
\usepackage{xpatch,filecontents}

\xpatchbibmacro{date+extrayear}{%
  \printtext[parens]%
}{%
  \setunit*{\addperiod\space}%
  \printtext%
}{}{}

% Supresses URL accessed day
\AtEveryBibitem{%
  \ifentrytype{electronic}
    {}
    {\clearfield{urlyear}\clearfield{urlmonth}\clearfield{urlday}}}
%\DefineBibliographyStrings{english}{%
%urlseen = {Accessed},}

\renewbibmacro*{volume+number+eid}{% number of journal in brackets
 \printfield{volume}%
  %\setunit*{\adddot}% DELETED
  \setunit*{\addnbthinspace}% NEW (optional); there's also \addnbthinspace
  \printfield{number}%
  \setunit{\addcomma\space}%
  \printfield{eid}}
\DeclareFieldFormat[article]{number}{\mkbibparens{#1}}



% Change edition field

\DeclareFieldFormat{edition}%
                   {\ifinteger{#1}%
                    {\mkbibordedition{#1}\addthinspace{}edition}%
                    {#1\isdot}}

% New command to show doi, or url or isbn or issn field
% http://tex.stackexchange.com/questions/48400/biblatex-make-title-hyperlink-to-dois-url-or-isbn
\newbibmacro{string+doiurlisbn}[1]{%
  \iffieldundef{doi}{%
    \iffieldundef{url}{%
      \iffieldundef{isbn}{%
        \iffieldundef{issn}{%
          #1%
        }{%
          \href{http://books.google.com/books?vid=ISSN\thefield{issn}}{#1}%
        }%
      }{%
        \href{http://books.google.com/books?vid=ISBN\thefield{isbn}}{#1}%
      }%
    }{%
      \href{\thefield{url}}{#1}%
    }%
  }{%
    \href{http://dx.doi.org/\thefield{doi}}{#1}%
  }%
}

% Necessary to remove dot after question mark in title
%\newcommand{\killpunct}[1]{}    

% Make full stop after title and before quotation marks in title field
\DeclareFieldFormat{title}{\usebibmacro{string+doiurlisbn}{\mkbibemph{#1}}}
\DeclareFieldFormat[article,incollection,unpublished,phdthesis]{title}%
    {\usebibmacro{string+doiurlisbn}{\mkbibquote{#1}}}
   % {\usebibmacro{string+doiurlisbn}{\mkbibquote{#1.\isdot}}}

\renewcommand*{\newunitpunct}{.\space}


\bibliography{/Users/smueller/Documents/GitHub/literature/muellerlibrary.bib}

\setcounter{secnumdepth}{0}


\usepackage{titlesec}
\titleformat{\subsubsection}
   {\normalfont\normalsize\itshape}{\thesubsubsection}{1em}{}
   
   
\usepackage{xcolor}
\definecolor{JournalBlue}{RGB}{0, 12, 146}
%https://en.wikibooks.org/wiki/LaTeX/Colors
\usepackage[colorlinks=true, linkcolor=JournalBlue, filecolor=black, urlcolor=JournalBlue, pdfborder={0 0 0},citecolor=JournalBlue]{hyperref}%RoyalBlue
%\usepackage[colorlinks]{hyperref}

\makeatletter
\def\blx@maxline{77}
\makeatother

\setlength\parindent{0pt}


\begin{document}
	
\singlespacing

\noindent
\adjustbox{valign=t}{\begin{minipage}{0.38\textwidth}% adapt widths of minipages to your needs
\includegraphics[width=\linewidth]{pictures/uzh_logo_de}
\end{minipage}}%
\hfill%
\adjustbox{valign=t}{\begin{minipage}{0.62\textwidth}\raggedleft
{%\footnotesize
\textbf{Stefan Müller} \\
Assistent \\
Lehrstuhl für Policy-Analyse \\
Universität Zürich \\
%\Letter\ \href{mailto:mullers@tcd.ie}{\textsf{mullers@tcd.ie}} \\
\url{https://muellerstefan.net} \\
}
\end{minipage}}

\singlespacing
\vspace{1cm}

\begin{center}
{\large Wahlmodul \href{https://studentservices.uzh.ch/uzh/anonym/vvz/index.html#/details/2018/004/E/50926420}{615251}} \\ 
\medskip
{\Large \textbf{Repräsentation, Parteienwettbewerb und Policy-Analyse}} \\
\bigskip
{\large  Entwurf (letzte Aktualisierung: \today)}

Neueste Version: \url{https://muellerstefan.net/teaching/2019-spring-rpp.pdf}
\end{center}

\vspace{1.5cm}

\hrule
\medskip
% first column
\begin{minipage}[t]{0.5\textwidth}
Semester: Frühjahrssemester 2019 \\
Zeit: Mittwoch, 10:15--11:45 \\
Veranstaltungsraum: \\
ECTS: 6.0
\end{minipage}
%second column
\begin{minipage}[t]{0.5\textwidth}
\begin{flushright}
Dozent: Stefan Müller \\
Raum: \\
Sprechstunde: \\
E-Mail: \textsf{\href{mailto:mueller@ipz.uzh.ch}{mueller@ipz.uzh.ch}}
\end{flushright}
\end{minipage}
\medskip
\vspace{2.5mm}
\hrule 

\section*{Kursbeschreibung}

Inwiefern unterscheiden sich Parteien inhaltlich? Erfüllen Parteien ihre Wahlversprechen? Unter welchen Umständen reagieren PolitikerInnen auf Änderungen in der öffentlichen Meinung? Und lernen Parteien voneinander? Antworten auf diese Fragen sind unverzichtbar, um gegenwärtige politische Debatten verstehen und einordnen zu können. Dieser Kurs ermöglicht einen systematischen Einblick in die wissenschaftliche Literatur über den Parteienwettbewerb, Repräsentationsfunktionen und öffentliche Politiken. Zu Beginn des Seminars werden zentrale Begriffe wie Repräsentation, Responsivität, Parteienwettbewerb, Wahlversprechen und Diffusion definiert und verknüpft. Daraufhin werden die Beziehungen zwischen Partien und WählerInnen in den Blick genommen, ehe wir uns der Frage widmen, inwieweit Parteipositionen und konkrete Politiken von Diffusionsprozessen und Umfrageergebnissen beeinflusst werden. Außerdem werden wir untersuchen, wie bestehende Studien diese Begriffe konzeptualisieren und welche Schwierigkeiten sich bei der Messung derart vielschichtiger Konzepte ergeben können.

\subsection*{Details}

\begin{itemize}
\item BA-Kurs
\item  Kurssprache: Deutsch
\item Benotung: Impulsreferat (10\%); kritische Stellungnahme (20\%); Literaturbericht (70\%)
\end{itemize}



\section*{Einführende Literatur}

\begin{itemize}
\item \fullcite{powell00}.
\item \fullcite{knill15}.
\item \fullcite{gallagher11}.
\item \fullcite{soroka10}.
\item \fullcite{dalton11}.
\item \fullcite{volkens13}.
\end{itemize}

\section*{Lernziele}

\begin{enumerate}
\item Vertiefung der Kenntnisse von zentralen theoretischen Aspekten der Parteien- und Policy-Forschung
\item Detaillierte Kenntnisse von aktuellen wissenschaftlichen Debatten über Repräsentation, Parteienwettbewerb, Responsivität und öffentliche Politiken
\item  Kritisches Lesen, Diskutieren und Aneignen der Inhalte komplexer Fachliteratur und diverser methodischer Vorgehensweisen
\item Konzeption und Durchführung einer kritischen Stellungnahme und eines ausführlichen Literaturberichts
\end{enumerate}

\section*{Erwartungen}

\begin{itemize}
\item Das \textbf{Impulsreferat} (10\%) \dots
\item Die \textbf{kritische Stellungnahme} (20\%)
\item Der \textbf{ausführliche Literaturbereicht} (70\%) 
\end{itemize}

\tableofcontents

\section{I. Einführung und Überblick}


\subsection{1. Woche: Organisatorisches und Aufbau des Moduls}

\begin{itemize}
\item Erwartungen
\item Besprechung des Syllabus
\item Hinweise zu Referate und der Anfertigung von schriftlichen Arbeiten
\end{itemize}


\subsubsection*{Pflichtlektüre}
\begin{itemize}
\item \fullcite[Kapitel 1]{clarke18}.
\item \fullcite{best11}.
\item \fullcite{manin99}.
\end{itemize}



\subsection{2. Woche: Überblick}

\begin{itemize}
\item Wie hängen Repräsentation, Parteienwettbewerb und Policy-Analyse zusammen?
\item Was sind die wichtigsten Theorien?
\end{itemize}


\subsubsection*{Pflichtlektüre}
\begin{itemize}
\item \fullcite[Kapitel 1]{dalton11}.
\item \fullcite[Kapitel 1--2]{powell00}.
\end{itemize}


\section{II. Repräsentation}

\subsection{3. Woche: Die Mandatstheorie und Wahlversprechen}

\subsubsection*{Pflichtlektüre}
\begin{itemize}
\item \fullcite{mansbridge03}.
\item \fullcite{thomson17}.
\end{itemize}

\subsubsection*{Optional/Referate}
\begin{itemize}
\item \fullcite{royed96}.
\item \fullcite{brouard18}.
\end{itemize}


\subsection{4. Die Messung der Öffentlichen Meinung}

\subsubsection*{Pflichtlektüre}
\begin{itemize}
\item \fullcite{berinsky17}.
\item \fullcite{bowler17}.
\end{itemize}


\subsubsection*{Optional/Referate}
\begin{itemize}
\item \fullcite{chong07b}.
\end{itemize}


%\subsection{4. Woche: Conditional Representation}
%
%canes-wrone 15 table 1 (electoral cycle, salience)

%\subsubsection*{Pflichtlektüre}
%\begin{itemize}
%\item \fullcite{budge15}.
%\item \fullcite{horn17}.
%\end{itemize}



\subsection{5. Verantwortbarkeit und die Kosten des Regierens}

\subsubsection*{Pflichtlektüre}
\begin{itemize}
\item \fullcite[Kapitel 5]{achen16}.
\item \fullcite{fowler18}.
\item \fullcite{healy13}.
\end{itemize}


\subsubsection*{Optional/Referate}
\begin{itemize}
\item \fullcite{sances17}.
\item \fullcite{reif80}.
\item \fullcite{muellerlouwerse}.
\end{itemize}


\subsection{6. Woche: Responsivität}

\subsubsection*{Pflichtlektüre}
\begin{itemize}
\item \fullcite{eulau77}.
\item \fullcite{wlezien95}.
\item \fullcite{powell04b}.
\end{itemize}

\subsubsection*{Optional/Referate}
\begin{itemize}
\item \fullcite{page83}.
\item \fullcite{stimson95}.
\item \fullcite{kluever16}.
%\item \fullcite{soroka10}.
\end{itemize}



\section{III. Parteienwettbeweb}



\subsection{7. Woche: Die Messung von Parteipositionen}

\subsubsection*{Pflichtlektüre}
\begin{itemize}
\item \fullcite{laver14}.
\item \fullcite{hjorth15}.
\item \fullcite{mikhaylov12}.
\end{itemize}


\subsubsection*{Optional/Referate}
\begin{itemize}
\item \fullcite{laver00}.
\item \fullcite{budge13}.
\item \fullcite{benoit16}.
\item \fullcite{lowe11}.
\end{itemize}



\subsection{8. Woche: Parteienwettbewerb}

\subsubsection*{Pflichtlektüre}
\begin{itemize}
\item \fullcite{stokes63}.
%\item \fullcite{adams05}.
\end{itemize}

\subsubsection*{Optional/Referate}
\begin{itemize}
\item \fullcite{greenpedersen07}.
\item \fullcite{wagner14}.
\item \fullcite{spoon15}.
\end{itemize}


\subsection{9. Woche: Salienz und Kongruenz}

\subsubsection*{Pflichtlektüre}
\begin{itemize}
\item \fullcite{budge15}.
\item \fullcite{caneswrone15}.
%\item \fullcite{spoon14}.
\item \fullcite[Kapitel ?]{powell00}.
\end{itemize}

\subsubsection*{Optional/Referate}
\begin{itemize}
\item \fullcite{powell09}.
\item \fullcite{kluever16}.
\item \fullcite{horn17}.
\end{itemize}


\subsection{10. Änderungen von Parteipositionen}

\subsubsection*{Pflichtlektüre}
\begin{itemize}
%\item \fullcite[Kapitel 1--2]{soroka10}.
\item \fullcite{boehmelt16}.
\item \fullcite{wolkenstein}.
\item \fullcite{bischof19}.
\end{itemize}



\subsubsection*{Optional/Referate}
\begin{itemize}
\item \fullcite{adams09}.
\item \fullcite{adams11}.
\item \fullcite{schumacher15}.
\item \fullcite{abouchadi20}.
\end{itemize}



\section{IV. Policy-Analyse}

\subsection{11. Der Policy-Prozess}

\subsubsection*{Pflichtlektüre}
\begin{itemize}
\item \fullcite[Kapitel 1--2]{knill15}.
\end{itemize}
 


\subsection{12. Policy-Diffusion}

\subsubsection*{Pflichtlektüre}
\begin{itemize}
\item \fullcite{graham13}.
\item \fullcite{gilardi16}.
\item \fullcite{elkins05}.
\end{itemize}
 
 
\subsubsection*{Optional/Referate}
\begin{itemize}
\item \fullcite{brooks07b}.
 \item \fullcite{gilardi10}.
\end{itemize}

 


\sloppy
\renewcommand*{\bibfont}{\small}

\setlength{\bibitemsep}{0.2em} % increase space between references
\printbibliography

\bigskip

\begin{center}
Letzte Aktualisierung: \today
\end{center}


\end{document}


